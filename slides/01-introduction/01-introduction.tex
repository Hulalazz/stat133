\documentclass[12pt]{beamer}\usepackage[]{graphicx}\usepackage[]{color}
%% maxwidth is the original width if it is less than linewidth
%% otherwise use linewidth (to make sure the graphics do not exceed the margin)
\makeatletter
\def\maxwidth{ %
  \ifdim\Gin@nat@width>\linewidth
    \linewidth
  \else
    \Gin@nat@width
  \fi
}
\makeatother

\definecolor{fgcolor}{rgb}{0.345, 0.345, 0.345}
\newcommand{\hlnum}[1]{\textcolor[rgb]{0.686,0.059,0.569}{#1}}%
\newcommand{\hlstr}[1]{\textcolor[rgb]{0.192,0.494,0.8}{#1}}%
\newcommand{\hlcom}[1]{\textcolor[rgb]{0.678,0.584,0.686}{\textit{#1}}}%
\newcommand{\hlopt}[1]{\textcolor[rgb]{0,0,0}{#1}}%
\newcommand{\hlstd}[1]{\textcolor[rgb]{0.345,0.345,0.345}{#1}}%
\newcommand{\hlkwa}[1]{\textcolor[rgb]{0.161,0.373,0.58}{\textbf{#1}}}%
\newcommand{\hlkwb}[1]{\textcolor[rgb]{0.69,0.353,0.396}{#1}}%
\newcommand{\hlkwc}[1]{\textcolor[rgb]{0.333,0.667,0.333}{#1}}%
\newcommand{\hlkwd}[1]{\textcolor[rgb]{0.737,0.353,0.396}{\textbf{#1}}}%

\usepackage{framed}
\makeatletter
\newenvironment{kframe}{%
 \def\at@end@of@kframe{}%
 \ifinner\ifhmode%
  \def\at@end@of@kframe{\end{minipage}}%
  \begin{minipage}{\columnwidth}%
 \fi\fi%
 \def\FrameCommand##1{\hskip\@totalleftmargin \hskip-\fboxsep
 \colorbox{shadecolor}{##1}\hskip-\fboxsep
     % There is no \\@totalrightmargin, so:
     \hskip-\linewidth \hskip-\@totalleftmargin \hskip\columnwidth}%
 \MakeFramed {\advance\hsize-\width
   \@totalleftmargin\z@ \linewidth\hsize
   \@setminipage}}%
 {\par\unskip\endMakeFramed%
 \at@end@of@kframe}
\makeatother

\definecolor{shadecolor}{rgb}{.97, .97, .97}
\definecolor{messagecolor}{rgb}{0, 0, 0}
\definecolor{warningcolor}{rgb}{1, 0, 1}
\definecolor{errorcolor}{rgb}{1, 0, 0}
\newenvironment{knitrout}{}{} % an empty environment to be redefined in TeX

\usepackage{alltt}
\usepackage{graphicx}
\setbeameroption{hide notes}
\setbeamertemplate{note page}[plain]
\usepackage{listings}

% get rid of junk
\usetheme{default}
\usefonttheme[onlymath]{serif}
\beamertemplatenavigationsymbolsempty
\hypersetup{pdfpagemode=UseNone} % don't show bookmarks on initial view

% named colors
\definecolor{offwhite}{RGB}{255,250,240}
\definecolor{gray}{RGB}{155,155,155}

\ifx\notescolors\undefined % slides

  \definecolor{foreground}{RGB}{80,80,80}
  \definecolor{background}{RGB}{255,255,255}
  \definecolor{title}{RGB}{255,199,0}
  \definecolor{subtitle}{RGB}{89,132,212}
  \definecolor{hilit}{RGB}{248,117,79}
  \definecolor{vhilit}{RGB}{255,111,207}
  \definecolor{lolit}{RGB}{200,200,200}
  \definecolor{lit}{RGB}{255,199,0}
  \definecolor{mdlit}{RGB}{89,132,212}
  \definecolor{link}{RGB}{248,117,79}

\else % notes
  \definecolor{background}{RGB}{255,255,255}
  \definecolor{foreground}{RGB}{24,24,24}
  \definecolor{title}{RGB}{27,94,134}
  \definecolor{subtitle}{RGB}{22,175,124}
  \definecolor{hilit}{RGB}{122,0,128}
  \definecolor{vhilit}{RGB}{255,0,128}
  \definecolor{lolit}{RGB}{95,95,95}
\fi
\definecolor{nhilit}{RGB}{128,0,128}  % hilit color in notes
\definecolor{nvhilit}{RGB}{255,0,128} % vhilit for notes

\newcommand{\hilit}{\color{hilit}}
\newcommand{\vhilit}{\color{vhilit}}
\newcommand{\nhilit}{\color{nhilit}}
\newcommand{\nvhilit}{\color{nvhilit}}
\newcommand{\lit}{\color{lit}}
\newcommand{\mdlit}{\color{mdlit}}
\newcommand{\lolit}{\color{lolit}}

% use those colors
\setbeamercolor{titlelike}{fg=title}
\setbeamercolor{subtitle}{fg=subtitle}
\setbeamercolor{frametitle}{fg=gray}
\setbeamercolor{structure}{fg=subtitle}
\setbeamercolor{institute}{fg=lolit}
\setbeamercolor{normal text}{fg=foreground,bg=background}
%\setbeamercolor{item}{fg=foreground} % color of bullets
%\setbeamercolor{subitem}{fg=hilit}
%\setbeamercolor{itemize/enumerate subbody}{fg=lolit}
\setbeamertemplate{itemize subitem}{{\textendash}}
\setbeamerfont{itemize/enumerate subbody}{size=\footnotesize}
\setbeamerfont{itemize/enumerate subitem}{size=\footnotesize}

% center title of slides
\setbeamertemplate{blocks}[rounded]
\setbeamertemplate{frametitle}[default][center]
% margins
\setbeamersize{text margin left=25pt,text margin right=25pt}

% page number
\setbeamertemplate{footline}{%
    \raisebox{5pt}{\makebox[\paperwidth]{\hfill\makebox[20pt]{\lolit
          \scriptsize\insertframenumber}}}\hspace*{5pt}}

% add a bit of space at the top of the notes page
\addtobeamertemplate{note page}{\setlength{\parskip}{12pt}}

% default link color
\hypersetup{colorlinks, urlcolor={link}}

\ifx\notescolors\undefined % slides
  % set up listing environment
  \lstset{language=bash,
          basicstyle=\ttfamily\scriptsize,
          frame=single,
          commentstyle=,
          backgroundcolor=\color{darkgray},
          showspaces=false,
          showstringspaces=false
          }
\else % notes
  \lstset{language=bash,
          basicstyle=\ttfamily\scriptsize,
          frame=single,
          commentstyle=,
          backgroundcolor=\color{offwhite},
          showspaces=false,
          showstringspaces=false
          }
\fi

% a few macros
\newcommand{\code}[1]{\texttt{#1}}
\newcommand{\hicode}[1]{{\hilit \texttt{#1}}}
\newcommand{\bb}[1]{\begin{block}{#1}}
\newcommand{\eb}{\end{block}}
\newcommand{\bi}{\begin{itemize}}
%\newcommand{\bbi}{\vspace{24pt} \begin{itemize} \itemsep8pt}
\newcommand{\bbi}{\vspace{4pt} \begin{itemize} \itemsep8pt}
\newcommand{\ei}{\end{itemize}}
\newcommand{\bv}{\begin{verbatim}}
\newcommand{\ev}{\end{verbatim}}
\newcommand{\ig}{\includegraphics}
\newcommand{\subt}[1]{{\footnotesize \color{subtitle} {#1}}}
\newcommand{\ttsm}{\tt \small}
\newcommand{\ttfn}{\tt \footnotesize}
\newcommand{\figh}[2]{\centerline{\includegraphics[height=#2\textheight]{#1}}}
\newcommand{\figw}[2]{\centerline{\includegraphics[width=#2\textwidth]{#1}}}



%------------------------------------------------
% end of header
%------------------------------------------------

\title{Course Introduction}
\subtitle{Stat 133 - Concepts in Computing with Data}
\author{\href{http://www.gastonsanchez.com}{Gaston Sanchez}}
\institute{Department of Statistics, UC{\textendash}Berkeley}
\date{\href{http://www.gastonsanchez.com}{\tt \scriptsize \color{foreground} gastonsanchez.com}
\\[-4pt]
\href{http://github.com/gastonstat/133}{\tt \scriptsize \color{foreground} github.com/gastonstat/stat133}
\\[-4pt]
{\scriptsize Course web: \href{http://www.gastonsanchez.com/stat133}{\tt gastonsanchez.com/stat133}}
}
\IfFileExists{upquote.sty}{\usepackage{upquote}}{}
\begin{document}

{
  \setbeamertemplate{footline}{} % no page number here
  \frame{
    \titlepage
  } 
}

%------------------------------------------------

\begin{frame}
\begin{center}
\Huge{\hilit{Concepts in Computing \\ with Data}}
\end{center}
\end{frame}

%------------------------------------------------

\begin{frame}[fragile]
\frametitle{}
\begin{center}
\ig[width=6cm]{images/my_name.pdf}
\end{center}
\end{frame}

%------------------------------------------------

\begin{frame}
\frametitle{A little bit about me}

\bbi
 \item Originally from Mexico
 \item Applied Statistician
 \item Statistical Programmer
 \item Data Analyst - Consultant
 \item Lecturer
\ei

\end{frame}

%------------------------------------------------

\begin{frame}
\begin{center}
\ig[width=11cm]{images/bikeuser.pdf}
\end{center}
\end{frame}

%------------------------------------------------

\begin{frame}
\begin{center}
\ig[width=11cm]{images/bikemaker.pdf}
\end{center}
\end{frame}

%------------------------------------------------

\begin{frame}
\begin{center}
{\Large \hilit{Concepts in Computing with Data?}}
\pause

\bigskip
{\Large \mdlit{Computational Data Analysis}}
\end{center}
\end{frame}

%------------------------------------------------

\begin{frame}
\begin{center}
\Huge{\mdlit{How to use computational tools to conduct a statistical analysis of data}}

\pause
\bigskip
\Huge{\lit{And thinking about Data Analysis}}

\end{center}
\end{frame}

%------------------------------------------------

\begin{frame}
\frametitle{Data Analysis}

\Large ``Data Analysis is the process by which data becomes understanding, knowledge and insight'' \\
{\lit \small{Hadley Wickham}}

\end{frame}

%------------------------------------------------

\begin{frame}
\frametitle{Data Analysis}

\bi
  \item Use the computer expressively to conduct statistical analysis of data
  \item Use existing software rather than build routines from the ground up
  \item Focus on aspects of computing to conduct statistical analysis, NOT the computational aspects of statistical methods
\ei

\end{frame}

%------------------------------------------------

\begin{frame}
\begin{center}
\Huge{\hilit{Quick Questions}}
\end{center}
\end{frame}

%------------------------------------------------

\begin{frame}
\frametitle{Major}

\bb{How many of you are}
\bbi
 \item Stats / Math majors?
 \item Non-stats majors?
 \item Double majors?
\ei
\eb

\end{frame}

%------------------------------------------------

\begin{frame}
\frametitle{Data Analysis Experience}

\bb{What's your data analysis experience?}
\bbi
 \item I'm completely new to data analysis
 \item I've analyzed data in Excel
 \item I've used statistical software (SAS, SPSS, etc)
\ei
\eb

\end{frame}

%------------------------------------------------

\begin{frame}
\frametitle{Programming Experience}

\bb{What's your programming experience?}
\bbi
 \item I have no programming experience
 \item I have some programming experience
 \item I've written some scripts in R, Python, Matlab
 \item I've used the command line (shell or terminal)
\ei
\eb

\end{frame}

%------------------------------------------------

\begin{frame}
\frametitle{Writing Documents}

\bb{When writing Docs and Reports}
\bbi
 \item I use word processors: \\
 Word, Pages, Google Docs, etc
 \item I use typesetting systems: \\
 LaTeX, markdown + pandoc
\ei
\eb

\end{frame}

%------------------------------------------------

\begin{frame}
\frametitle{Preparing Slides}

\bb{When preparing slides}
\bbi
 \item I use Power Point, Keynote, Google Slides
 \item I use LaTeX beamer, markdown, HTML5
\ei
\eb

\end{frame}

%------------------------------------------------

\begin{frame}
\begin{center}
\Huge{\hilit{Syllabus}}
\end{center}
\end{frame}

%------------------------------------------------

\begin{frame}
\frametitle{Course webpage}
\begin{center}
\Large{gastonsanchez.com/teaching/stat133}
\end{center}
\end{frame}

%------------------------------------------------

\begin{frame}
\frametitle{Tentative Content}

\bi
 \item R Basics
 \item Statistical Graphics
 \item Exploratory Data Analysis
 \item Programming in R
 \item Regular Expressions
 \item Data Manipulation
 \item Data Technologies
 \item Simulations
 \item Reporting and Communicating
\ei

\end{frame}

%------------------------------------------------

\begin{frame}
\frametitle{Course Info}

\bbi
 \item $\sim$ 15 weeks (Aug 26 - Dec 14)
 \item $\sim$ 45 hours of lecture
 \item $\sim$ 30 lab hours
 \item 3 units
\ei

\end{frame}

%------------------------------------------------

\begin{frame}
\frametitle{Course Work}

\bb{Course Work}
\bbi
 \item $\sim$ 8 Homework assignments
 \item $\sim$ 8 Lab assignments
 \item 2 Exams (midterm Oct-22/23, final Dec-14/15)
 \item 1 Individual project \\ 
 \item 1 Group project \\
{\lit more about the projects in the next weeks}
\ei
\eb

\end{frame}

%------------------------------------------------

\begin{frame}
\frametitle{Overall Score}

\begin{center}
 \begin{tabular}{l c}
  \hline
  Value & Concept \\
  \hline
  10\% & Participation \\
  15\% & Lab \\
  25\% & Homework \\
  25\% & Midterm \& Final exam \\
  25\% & Projects \\
  \hline
 \end{tabular}
\end{center}

\end{frame}

%------------------------------------------------

\begin{frame}
\frametitle{Attendance Policy}

\begin{center}
\Huge{\hilit{You are expected to attend all lectures and lab discussions}}
\end{center}

\end{frame}

%------------------------------------------------

\begin{frame}
\frametitle{Office Hours}

To be announced ... or by appointment (preferred)

\end{frame}

%------------------------------------------------

\begin{frame}
\begin{center}
\Huge{\hilit{Course Policies}}
\end{center}
\end{frame}

%------------------------------------------------

\begin{frame}
\frametitle{}
\begin{center}
\ig[width=11cm]{images/dos-donts.pdf}
\end{center}
\end{frame}

%------------------------------------------------

\begin{frame}
\frametitle{}
\begin{center}
\ig[width=10cm]{images/nocellphones.pdf}
\end{center}
\end{frame}

%------------------------------------------------

\begin{frame}
\frametitle{}
\begin{center}
\ig[width=8cm]{images/laptop.jpg}
\end{center}
\end{frame}

%------------------------------------------------

\begin{frame}
\begin{center}
\ig[width=11cm]{images/pleasedont.pdf}
\end{center}
\end{frame}

%------------------------------------------------

\begin{frame}
\begin{center}
\ig[width=11cm]{images/honorcode.pdf}
\end{center}
\end{frame}

%------------------------------------------------

\begin{frame}
\frametitle{Academic Integrity}

\bi
 \item Write your own scripts and code
 \item You can share ideas and exchange suggestions, but code must be yours
 \item If you use someone else's code (e.g. find it online), give credit
 \item Plagiarism won't be tolerated
\ei

\end{frame}

%------------------------------------------------

\begin{frame}
\frametitle{Email Policy}

\begin{center}
\Huge No email
\end{center}

\end{frame}

%------------------------------------------------

\begin{frame}
\frametitle{Email}

\bb{Email Policy}
\bi
 \item Use email as a tool to set up a one-on-one meeting
 \item Use the subject line {\hilit Meeting Request}
 \item Include at least two times when you would like to meet 
 \item Include a brief (one-two sentence) description of the reason for the meeting
 \item Email sent for any other reason will NOT be considered or acknowledge
 \item Don't expect me to reply right away \\
 {\lolit I'm a data analyst, not a professional email responder}
\ei
\eb

\end{frame}

%------------------------------------------------

\begin{frame}
\frametitle{Let's talk}

\bb{Q \& A's}
I strongly encourage you to ask questions about the syllabus and assignments during class time. 

\bigskip
For more in-depth discussions (such as guidance on assignments) please plan to meet in person.
\eb

\end{frame}

%------------------------------------------------

\begin{frame}
\frametitle{Resources}

\bb{Course webpage}
  \url{http://gastonsanchez.com/teaching/stat133}
\eb

\bb{Github Repo}
  \url{https://github.com/gastonstat/stat133}
\eb

\end{frame}

%------------------------------------------------

\begin{frame}
\begin{center}
\Huge{\hilit{About this course}}
\end{center}
\end{frame}

%------------------------------------------------

\begin{frame}
\frametitle{Computing with Data (CwD)}

\bb{Computing with Data (CwD)}
\pause
\bbi
 \item CwD means everything and nothing at the same time
 \item Data Analysis
 \item Data Manipulation
 \item Statistical Programming
\ei
\eb

\end{frame}

%------------------------------------------------

\begin{frame}
\frametitle{Computing with Data}

\Large ``Computing with data refers to activities in which data is acquired, managed, and processed for a great variety of purposes: organization, visualization, summaries, analysis, etc'' \\
{\lit \small{John Chambers}}

\end{frame}

%------------------------------------------------

\begin{frame}
\frametitle{Computing with Data}

\Large ``Computing with data refers to activities in which data is \textbf{acquired}, \textbf{managed}, and \textbf{processed} for a great variety of purposes: \textbf{organization}, \textbf{visualization}, \textbf{summaries}, \textbf{analysis}, etc'' \\
{\lit \small{John Chambers}}

\end{frame}

%------------------------------------------------

\begin{frame}
\begin{center}
\Huge{\hilit{Understanding the Data Analysis Process}}
\end{center}
\end{frame}

%------------------------------------------------

\begin{frame}[fragile]
\begin{center}
\ig[width=11cm]{images/data-by-the-numbers.png}
\end{center}
\end{frame}

%------------------------------------------------

\begin{frame}[fragile]
\begin{center}
\ig[width=3cm]{images/databynumbers1.pdf}
\end{center}
\end{frame}

%------------------------------------------------

\begin{frame}[fragile]
\begin{center}
\ig[width=3cm]{images/databynumbers2.pdf}
\end{center}
\end{frame}

%------------------------------------------------

\begin{frame}[fragile]
\begin{center}
\ig[width=3cm]{images/databynumbers3.pdf}
\end{center}
\end{frame}

%------------------------------------------------

\begin{frame}[fragile]
\begin{center}
\ig[width=3cm]{images/databynumbers4.pdf}
\end{center}
\end{frame}

%------------------------------------------------

\begin{frame}
\begin{center}
\Huge{\hilit{Data Analysis Stages}}
\end{center}
\end{frame}

%------------------------------------------------

\begin{frame}
\frametitle{Workflow}

\bb{Data Workflow}
\begin{enumerate}
 \item Acquisition-collection
 \item Processing and Cleaning
 \item Exploration and Visualization
 \item Modeling
 \item Reporting and Communication
\end{enumerate}
\eb

\end{frame}

%------------------------------------------------

\begin{frame}
\frametitle{Main Stages of a Data Analysis Cycle}
\begin{center}
\ig[width=11cm]{images/analysis-cycle.pdf}

\bigskip
{\lit This is not a linear workflow; \\ there are iterative cycles at any stage}
\end{center}
\end{frame}

%------------------------------------------------

\begin{frame}[fragile]
\begin{center}
\ig[width=8cm]{images/theory-practice.pdf}
\end{center}
\end{frame}

%------------------------------------------------

\begin{frame}
\begin{center}
\Huge{\hilit{Data Analysis is a lot like cooking \& baking}}
\end{center}
\end{frame}

%------------------------------------------------

\begin{frame}
\frametitle{Work Environment}
\begin{center}
\ig[width=10.5cm]{images/kitchen.jpg}
\end{center}
\end{frame}

%------------------------------------------------

\begin{frame}
\frametitle{Tools \& Utensils}
\begin{center}
\ig[width=7cm]{images/tools.jpg}
\end{center}
\end{frame}

%------------------------------------------------

\begin{frame}
\frametitle{Raw Data}
\begin{center}
\ig[width=10cm]{images/raw-ingredients.jpg}
\end{center}
\end{frame}

%------------------------------------------------

\begin{frame}
\frametitle{Processing, Cleaning, Organizing}
\begin{center}
\ig[width=10cm]{images/ingredients.jpg}
\end{center}
\end{frame}

%------------------------------------------------

\begin{frame}
\frametitle{Manipulation}
\begin{center}
\ig[width=10cm]{images/dough.jpg}
\end{center}
\end{frame}

%------------------------------------------------

\begin{frame}
\frametitle{Modeling}
\begin{center}
\ig[width=10cm]{images/cherry-pie.jpg}
\end{center}
\end{frame}

%------------------------------------------------

\begin{frame}
\frametitle{Presentation \& Consumption}
\begin{center}
\ig[width=9cm]{images/pie-slice.jpg}
\end{center}
\end{frame}

%------------------------------------------------

\begin{frame}
\frametitle{Good -vs- Bad Practices}
\begin{center}
\ig[width=10cm]{images/kitchen-mess.jpg}
\end{center}
\end{frame}

%------------------------------------------------

\begin{frame}
\frametitle{Housekeeping}
\begin{center}
\ig[width=8.5cm]{images/washing.jpg}
\end{center}
\end{frame}

%------------------------------------------------

\begin{frame}
  \begin{center}
    \Huge{Becoming a data scientist \\ is a {\hilit \textbf{Marathon}}, not a sprint}
  \end{center}
\end{frame}

%------------------------------------------------

\begin{frame}
\begin{center}
\Huge{\hilit{Working Environment}}
\end{center}
\end{frame}

%------------------------------------------------

\begin{frame}
\frametitle{Tools}

\bb{Tools for this course}
\bi
 \item Computer
 \item Text Editor
 \item R (software for data analysis)
 \item RStudio (IDE)
 \item Terminal or Command Line
 \item Latex
 \item git*
 \item github account*
\ei
\eb

\end{frame}

%------------------------------------------------

\begin{frame}
\frametitle{Tools}

\bb{Text editors for OS X, Windows and Linux}
Choose a text editor of your preference:
\bi
 \item Emacs:
 {\scriptsize \url{http://www.gnu.org/software/emacs/}}
 \item Atom: 
 {\scriptsize \url{https://atom.io/}}
 \item Sublime Text: 
 {\scriptsize \url{http://www.sublimetext.com/}}
 \item Vim 
 {\scriptsize \url{http://www.vim.org/download.php}}
 \item TextWrangler (Mac OS X only): 
 {\scriptsize \url{http://www.barebones.com/products/textwrangler/}}
 \item Notepad ++ (Windows only): 
 {\scriptsize \url{https://notepad-plus-plus.org/}}
\ei
\eb

\end{frame}

%------------------------------------------------

\begin{frame}
\begin{center}
\ig[width=11cm]{images/r-website.png}
\end{center}
\end{frame}

%------------------------------------------------

\begin{frame}
\frametitle{Getting R}

\bb{Download R}
\bi
 \item R project website \\
 \url{http://www.r-project.org}
 \item Go to CRAN \\
 \url{http://cran.r-project.org/mirrors.html}
 \item Select closest mirror, e.g. \\
 \url{http://cran.cnr.Berkeley.edu/}
 \item Choose the right version for your Operating System \\
 {\lolit (e.g. Linux, Mac, Windows)}
\ei
\eb

\end{frame}

%------------------------------------------------

\begin{frame}
\frametitle{Getting RStudio}

\bb{Download RStudio}
\bi
 \item Rstudio \\
 \url{http://www.rstudio.com/}
 \item RStudio desktop \\
 \small{\url{http://www.rstudio.com/products/rstudio/download/}}
 \item Choose the right version for your Operating System \\
 {\lolit (e.g. Linux, Mac, Windows)}
\ei
\eb

\end{frame}

%------------------------------------------------

\begin{frame}
\frametitle{Tools}

\bb{Rtools (for Windows only)}
If you work with Windows, you'll need \textbf{Rtools}:
 {\scriptsize \url{http://cran.r-project.org/bin/windows/Rtools/}}
\eb

\bb{Command Line Tools (Mac)}
If you've never used the Mac Terminal before, it's likely that you'll need to install the {\lit Command Line Tools} (you'll know if you need this).
\eb

\end{frame}

%------------------------------------------------

\begin{frame}
\frametitle{Tools}

\bb{LaTeX}
If you don't have it, install LaTeX:
\bi
 \item TeX Live (Linux):
 {\scriptsize \url{http://www.tug.org/texlive/}}
 \item MacTeX (Mac OS X): 
 {\scriptsize \url{http://www.tug.org/mactex/}}
 \item ProTeXt (Windows): 
 {\scriptsize \url{http://www.tug.org/protext/}}
\ei
\eb

\end{frame}

%------------------------------------------------

\begin{frame}
\frametitle{Tools}

\bb{git}
For all platforms:
 {\scriptsize \url{http://git-scm.com/book/en/v2/Getting-Started-Installing-Git}}
\eb

\bb{github}
If you don't have it yet, open a (free) account in github
 {\scriptsize \url{https://github.com/join}}
\eb

\end{frame}


%------------------------------------------------

\end{document}
