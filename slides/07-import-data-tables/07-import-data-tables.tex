\documentclass[12pt]{beamer}\usepackage[]{graphicx}\usepackage[]{color}
%% maxwidth is the original width if it is less than linewidth
%% otherwise use linewidth (to make sure the graphics do not exceed the margin)
\makeatletter
\def\maxwidth{ %
  \ifdim\Gin@nat@width>\linewidth
    \linewidth
  \else
    \Gin@nat@width
  \fi
}
\makeatother

\definecolor{fgcolor}{rgb}{0.345, 0.345, 0.345}
\newcommand{\hlnum}[1]{\textcolor[rgb]{0.686,0.059,0.569}{#1}}%
\newcommand{\hlstr}[1]{\textcolor[rgb]{0.192,0.494,0.8}{#1}}%
\newcommand{\hlcom}[1]{\textcolor[rgb]{0.678,0.584,0.686}{\textit{#1}}}%
\newcommand{\hlopt}[1]{\textcolor[rgb]{0,0,0}{#1}}%
\newcommand{\hlstd}[1]{\textcolor[rgb]{0.345,0.345,0.345}{#1}}%
\newcommand{\hlkwa}[1]{\textcolor[rgb]{0.161,0.373,0.58}{\textbf{#1}}}%
\newcommand{\hlkwb}[1]{\textcolor[rgb]{0.69,0.353,0.396}{#1}}%
\newcommand{\hlkwc}[1]{\textcolor[rgb]{0.333,0.667,0.333}{#1}}%
\newcommand{\hlkwd}[1]{\textcolor[rgb]{0.737,0.353,0.396}{\textbf{#1}}}%

\usepackage{framed}
\makeatletter
\newenvironment{kframe}{%
 \def\at@end@of@kframe{}%
 \ifinner\ifhmode%
  \def\at@end@of@kframe{\end{minipage}}%
  \begin{minipage}{\columnwidth}%
 \fi\fi%
 \def\FrameCommand##1{\hskip\@totalleftmargin \hskip-\fboxsep
 \colorbox{shadecolor}{##1}\hskip-\fboxsep
     % There is no \\@totalrightmargin, so:
     \hskip-\linewidth \hskip-\@totalleftmargin \hskip\columnwidth}%
 \MakeFramed {\advance\hsize-\width
   \@totalleftmargin\z@ \linewidth\hsize
   \@setminipage}}%
 {\par\unskip\endMakeFramed%
 \at@end@of@kframe}
\makeatother

\definecolor{shadecolor}{rgb}{.97, .97, .97}
\definecolor{messagecolor}{rgb}{0, 0, 0}
\definecolor{warningcolor}{rgb}{1, 0, 1}
\definecolor{errorcolor}{rgb}{1, 0, 0}
\newenvironment{knitrout}{}{} % an empty environment to be redefined in TeX

\usepackage{alltt}
\usepackage{graphicx}
\usepackage{tikz}
\setbeameroption{hide notes}
\setbeamertemplate{note page}[plain]
\usepackage{listings}

% get rid of junk
\usetheme{default}
\usefonttheme[onlymath]{serif}
\beamertemplatenavigationsymbolsempty
\hypersetup{pdfpagemode=UseNone} % don't show bookmarks on initial view

% named colors
\definecolor{offwhite}{RGB}{255,250,240}
\definecolor{gray}{RGB}{155,155,155}

\ifx\notescolors\undefined % slides

  \definecolor{foreground}{RGB}{80,80,80}
  \definecolor{background}{RGB}{255,255,255}
  \definecolor{title}{RGB}{255,199,0}
  \definecolor{subtitle}{RGB}{89,132,212}
  \definecolor{hilit}{RGB}{248,117,79}
  \definecolor{vhilit}{RGB}{255,111,207}
  \definecolor{lolit}{RGB}{200,200,200}
  \definecolor{lit}{RGB}{255,199,0}
  \definecolor{mdlit}{RGB}{89,132,212}
  \definecolor{link}{RGB}{248,117,79}

\else % notes
  \definecolor{background}{RGB}{255,255,255}
  \definecolor{foreground}{RGB}{24,24,24}
  \definecolor{title}{RGB}{27,94,134}
  \definecolor{subtitle}{RGB}{22,175,124}
  \definecolor{hilit}{RGB}{122,0,128}
  \definecolor{vhilit}{RGB}{255,0,128}
  \definecolor{lolit}{RGB}{95,95,95}
\fi
\definecolor{nhilit}{RGB}{128,0,128}  % hilit color in notes
\definecolor{nvhilit}{RGB}{255,0,128} % vhilit for notes

\newcommand{\hilit}{\color{hilit}}
\newcommand{\vhilit}{\color{vhilit}}
\newcommand{\nhilit}{\color{nhilit}}
\newcommand{\nvhilit}{\color{nvhilit}}
\newcommand{\lit}{\color{lit}}
\newcommand{\mdlit}{\color{mdlit}}
\newcommand{\lolit}{\color{lolit}}

% use those colors
\setbeamercolor{titlelike}{fg=title}
\setbeamercolor{subtitle}{fg=subtitle}
\setbeamercolor{frametitle}{fg=gray}
\setbeamercolor{structure}{fg=subtitle}
\setbeamercolor{institute}{fg=lolit}
\setbeamercolor{normal text}{fg=foreground,bg=background}
%\setbeamercolor{item}{fg=foreground} % color of bullets
%\setbeamercolor{subitem}{fg=hilit}
%\setbeamercolor{itemize/enumerate subbody}{fg=lolit}
\setbeamertemplate{itemize subitem}{{\textendash}}
\setbeamerfont{itemize/enumerate subbody}{size=\footnotesize}
\setbeamerfont{itemize/enumerate subitem}{size=\footnotesize}

% center title of slides
\setbeamertemplate{blocks}[rounded]
\setbeamertemplate{frametitle}[default][center]
% margins
\setbeamersize{text margin left=25pt,text margin right=25pt}

% page number
\setbeamertemplate{footline}{%
    \raisebox{5pt}{\makebox[\paperwidth]{\hfill\makebox[20pt]{\lolit
          \scriptsize\insertframenumber}}}\hspace*{5pt}}

% add a bit of space at the top of the notes page
\addtobeamertemplate{note page}{\setlength{\parskip}{12pt}}

% default link color
\hypersetup{colorlinks, urlcolor={link}}

\ifx\notescolors\undefined % slides
  % set up listing environment
  \lstset{language=bash,
          basicstyle=\ttfamily\scriptsize,
          frame=single,
          commentstyle=,
          backgroundcolor=\color{darkgray},
          showspaces=false,
          showstringspaces=false
          }
\else % notes
  \lstset{language=bash,
          basicstyle=\ttfamily\scriptsize,
          frame=single,
          commentstyle=,
          backgroundcolor=\color{offwhite},
          showspaces=false,
          showstringspaces=false
          }
\fi

% a few macros
\newcommand{\code}[1]{\texttt{#1}}
\newcommand{\hicode}[1]{{\hilit \texttt{#1}}}
\newcommand{\bb}[1]{\begin{block}{#1}}
\newcommand{\eb}{\end{block}}
\newcommand{\bi}{\begin{itemize}}
%\newcommand{\bbi}{\vspace{24pt} \begin{itemize} \itemsep8pt}
\newcommand{\bbi}{\vspace{4pt} \begin{itemize} \itemsep8pt}
\newcommand{\ei}{\end{itemize}}
\newcommand{\bv}{\begin{verbatim}}
\newcommand{\ev}{\end{verbatim}}
\newcommand{\ig}{\includegraphics}
\newcommand{\subt}[1]{{\footnotesize \color{subtitle} {#1}}}
\newcommand{\ttsm}{\tt \small}
\newcommand{\ttfn}{\tt \footnotesize}
\newcommand{\figh}[2]{\centerline{\includegraphics[height=#2\textheight]{#1}}}
\newcommand{\figw}[2]{\centerline{\includegraphics[width=#2\textwidth]{#1}}}



%------------------------------------------------
% end of header
%------------------------------------------------

\title{Reading Data Tables}
\subtitle{STAT 133}
\author{\href{http://www.gastonsanchez.com}{Gaston Sanchez}}
\institute{Department of Statistics, UC{\textendash}Berkeley}
\date{\href{http://www.gastonsanchez.com}{\tt \scriptsize \color{foreground} gastonsanchez.com}
\\[-4pt]
\href{http://github.com/gastonstat/stat133}{\tt \scriptsize \color{foreground} github.com/gastonstat/stat133}
\\[-4pt]
{\scriptsize Course web: \href{http://www.gastonsanchez.com/stat133}{\tt gastonsanchez.com/stat133}}
}
\IfFileExists{upquote.sty}{\usepackage{upquote}}{}
\begin{document}


{
  \setbeamertemplate{footline}{} % no page number here
  \frame{
    \titlepage
  } 
}

%------------------------------------------------

\begin{frame}
\begin{center}
\Huge{\hilit{So far ...}}
\end{center}
\end{frame}

%------------------------------------------------

\begin{frame}
\frametitle{So far}

\bi
  \item Data Structures in R
  \bi
    \item Vectors and Factors
    \item Matrices and Arrays
    \item Data Frames and Lists
  \ei
  \item Emphasis on \textbf{vectors}
  \item Atomic -vs- Non-atomic objects
  \item Vectorization
  \item Recycling
  \item Bracket Notation
\ei

\end{frame}

%------------------------------------------------

\begin{frame}
\begin{center}
\Huge{\hilit{Datasets}}
\end{center}
\end{frame}

%------------------------------------------------

\begin{frame}
\frametitle{Datasets}

\bb{You'll have some sort of (raw) data to work with}
\eb
\begin{center}
\ig[width=10cm]{images/tabular_nontabular.pdf}
\end{center}

\end{frame}

%------------------------------------------------

\begin{frame}[fragile]
\frametitle{Some Data}
\begin{center}
\ig[width=10cm]{images/leia_luke_han.pdf}
\end{center}
\end{frame}

%------------------------------------------------

\begin{frame}[fragile]
\frametitle{Toy Data (tabular layout)}

\begin{center}
 \begin{tabular}{| l | l | l |}
  \hline
name & gender & height \\
  \hline
Leia Skywalker & female & 1.50 \\
  \hline
Luke Skywalker & male & 1.72 \\
  \hline
Han Solo & male & 1.80 \\
  \hline
 \end{tabular}
\end{center}

\end{frame}

%------------------------------------------------

\begin{frame}
\frametitle{Data Table (conceptually)}

\bbi
  \item Conceptually (and visually), tabular data consists of a rectangular array of cells
  \item Tables have rows and columns
  \item Intersection of row and column gives a cell
  \item A data value lies in each table cell
\ei

\end{frame}

%------------------------------------------------

\begin{frame}
\begin{center}
\Large{\mdlit{Data can also be \\ in non-tabular format}}

\end{center}
\end{frame}

%------------------------------------------------

\begin{frame}[fragile]
\frametitle{Toy Data (XML format)}

\begin{knitrout}\footnotesize
\definecolor{shadecolor}{rgb}{0.969, 0.969, 0.969}\color{fgcolor}\begin{kframe}
\begin{alltt}
<subject>
  <name>Leia Skywalker</name>
  <gender>female</gender>
  <height>1.50</height>
</subject>
<subject>
  <name>Luke Skywalker</name>
  <gender>male</gender>
  <height>1.72</height>
</subject>
<subject>
  <name>Han Solo</name>
  <gender>male</gender>
  <height>1.80</height>
</subject>
\end{alltt}
\end{kframe}
\end{knitrout}

\end{frame}

%------------------------------------------------

\begin{frame}[fragile]
\frametitle{Toy Data (JSON format)}

\begin{knitrout}\footnotesize
\definecolor{shadecolor}{rgb}{0.969, 0.969, 0.969}\color{fgcolor}\begin{kframe}
\begin{alltt}
\{
  \hlstr{"subject"} : \{
    \hlstr{"name"} : \hlstr{"Leia Skywalker"},
    \hlstr{"gender"} : \hlstr{"female"},
    \hlstr{"height"} : 1.50
  \},
  \hlstr{"subject"} : \{
    \hlstr{"name"} : \hlstr{"Luke Skywalker"},
    \hlstr{"gender"} : \hlstr{"male"},
    \hlstr{"height"} : 1.72
  \},
  \hlstr{"subject"} : \{
    \hlstr{"name"} : \hlstr{"Han Solo"},
    \hlstr{"gender"} : \hlstr{"male"},
    \hlstr{"height"} : 1.80
  \}
\}
\end{alltt}
\end{kframe}
\end{knitrout}

\end{frame}

%------------------------------------------------

\begin{frame}[fragile]
\frametitle{Toy Data (other format)}

\begin{knitrout}\footnotesize
\definecolor{shadecolor}{rgb}{0.969, 0.969, 0.969}\color{fgcolor}\begin{kframe}
\begin{alltt}
\hlstr{"Leia Skywalker"}
\hlstd{gender}\hlopt{:} \hlstd{female}
\hlstd{height}\hlopt{:} \hlnum{1.50}

\hlstr{"Luke Skywalker"}
\hlstd{gender}\hlopt{:} \hlstd{male}
\hlstd{height}\hlopt{:} \hlnum{1.72}

\hlstr{"Han Solo"}
\hlstd{gender}\hlopt{:} \hlstd{male}
\hlstd{height}\hlopt{:} \hlnum{1.80}
\end{alltt}
\end{kframe}
\end{knitrout}

\end{frame}

%------------------------------------------------

\begin{frame}[fragile]
\frametitle{Toy Data (other format)}

\begin{knitrout}\footnotesize
\definecolor{shadecolor}{rgb}{0.969, 0.969, 0.969}\color{fgcolor}\begin{kframe}
\begin{alltt}
Leia Skywalker
F 1.50
***
Luke Skywalker
M 1.72
***
Han Solo
M 1.80
\end{alltt}
\end{kframe}
\end{knitrout}

\end{frame}

%------------------------------------------------

\begin{frame}
\frametitle{Data Tables}

Many datasets come in tabular form: rectangular array of rows and columns (e.g. spreadsheet)

\begin{center}
\ig[height=3cm]{images/tablegrid.pdf}
\end{center}

{\small
In this lecture we'll focus on how to read this type of data in R (we'll talk about how to read other types of datasets in a different lecture)
}

\end{frame}

%------------------------------------------------

\begin{frame}[fragile]
\frametitle{Data Tables}

\begin{center}
{\Large How to store tables in a file?}

\bigskip

 \begin{tabular}{| l | l | l |}
  \hline
name & gender & height \\
  \hline
Leia Skywalker & female & 1.50 \\
  \hline
Luke Skywalker & male & 1.72 \\
  \hline
Han Solo & male & 1.80 \\
  \hline
 \end{tabular}
\end{center}

\end{frame}

%------------------------------------------------

\begin{frame}
\begin{center}
\Huge{\hilit{Files and Memory}}
\end{center}
\end{frame}

%------------------------------------------------

\begin{frame}[fragile]
\frametitle{}
\begin{center}
\ig[width=10cm]{images/tabular_nontabular_files.pdf}
\end{center}
\end{frame}

%------------------------------------------------

\begin{frame}
\frametitle{Files and Formats}

\bbi
  \item We store Data Sets in files
  \item A \textbf{file} is simply a block of computer memory
  \item A file can be as small as just a few bytes or it can be several gigabytes in size (thousands of millions of bytes)
\ei

\end{frame}

%------------------------------------------------

\begin{frame}
\frametitle{BIT}

\bbi
  \item The most fundamental unit of computer memory is the \textbf{bit}
  \bi
    \item can be a tiny magnetic region on a hard disk
    \item can be a tiny transistor on a memory disk
    \item can be a tiny dent in the reflective material on a CD or DVD
  \ei
  \item A bit is like a {\hilit switch}, it can only take two values: 
  \bi
    \item \textbf{on} (1)
    \item \textbf{off} (0)
  \ei
  \item A bit is a single \textbf{binary digit} (0 or 1)
\ei

\end{frame}

%------------------------------------------------

\begin{frame}
\frametitle{Binary Digit}

\bbi
  \item All computers are binary (0, 1)
  \item Binary code is used to store everything
  \bi
    \item numbers: 0, 1, -30, 3.1416, ...
    \item characters: a, \$, ), ...
    \item instructions: \code{sum}, \code{sqrt}, ...
    \item colors: \textit{red, green, blue}, ...
  \ei
\ei

\end{frame}

%------------------------------------------------

\begin{frame}
\frametitle{Representing Numbers}

Recall that when we write a 3-digit number, e.g. \\
\begin{center}
{\large \textbf{105}}
\end{center}

\pause
we are using the decimal system:
\bi
  \item \textbf{1} hundreds
  \item \textbf{0} tens
  \item \textbf{5} ones
\ei

that is: $(1 \times 10^2) + (0 \times 10^1) + (5 \times 10^0)$ \\
{\footnotesize where the digits range 0, 1, 2, ..., 9}

\end{frame}

%------------------------------------------------

\begin{frame}
\frametitle{Representing Numbers in Binary}

The binary number \\
\begin{center}
{\large \textbf{1101001}}
\end{center}

\pause
now we have powers of 2 and digits 0 and 1 \\

\bigskip

$(1 \times 2^6) + (1 \times 2^5) + (0 \times 2^4) + 
(1 \times 2^3) + (0 \times 2^2) + (0 \times 2^1) + (1 \times 2^0)$

\pause
\bigskip
In decimal digits this is: 64 + 32 + 8 + 1 = 105
\end{frame}

%------------------------------------------------

\begin{frame}
\frametitle{Representing Numbers in Binary}

{\hilit \textbf{Clicker:}} What is the decimal value of the following 4-digit binary number \\
\begin{center}
{\large \textbf{1110}}
\end{center}

\bi
  \item A: 5
  \item B: 8
  \item C: 14
  \item D: 12
\ei

\pause
\bigskip

$(1 \times 2^3) + (1 \times 2^2) + (1 \times 2^1) + (0 \times 2^0)$

\pause
\bigskip
8 + 4 + 2 + 0 = \textbf{14}
\end{frame}

%------------------------------------------------

\begin{frame}[fragile]
\frametitle{BITS}
\begin{center}
\ig[width=9cm]{images/bits.pdf}
\bigskip

{\tiny Each additional bit doubles the number of possible permutations. $N$ bits represent values 0 to $2^{N-1}$}
\end{center}
\end{frame}

%------------------------------------------------

\begin{frame}
\frametitle{Bits and Bytes}

\bbi
  \item A collection of 8 bits is a \textbf{byte}
  \item Each byte can store:
  \bi
    \item numbers: 00000000 (0), to 11111111 (255)
    \item has a memory address: 0, 1, 2, ...
  \ei
  \item To store bigger numbers, we use several bytes
  \bi
    \item 2 bytes: 0 to 65,535
    \item 4 bytes: 0 to 4,294,967,295
    \item 4 bytes (1 byte for $\pm$): $\pm$ 2,147,483,648
  \ei
  \item Every memory device has a storage capacity indicating the number of bytes it can hold
\ei

\end{frame}

%------------------------------------------------

\begin{frame}
\frametitle{Files and Formats}

\begin{center}
\ig[width=5cm]{images/binary_file_icon.jpg}
\end{center}
Every file is binary in the sense that it consists of 0s and 1s

\end{frame}

%------------------------------------------------

\begin{frame}
\frametitle{Files and Formats}

\bb{A file format:}
\bbi
  \item is a way of interpreting the bytes in a file
  \item specifies how bits are used to encode information in a digital storage medium
  \item For example, in the simplest case, a \textbf{plain text} format means that each byte is used to represent a single character
\ei
\eb

\end{frame}

%------------------------------------------------

\begin{frame}
\frametitle{Some Confusing Terms}

\bbi
  \item Text files
  \item Plain text files
  \item Formatted text files
  \item Enriched text files
\ei

\end{frame}

%------------------------------------------------

\begin{frame}
\frametitle{Some Confusing Terms}

Let's take the term \textbf{text files} to mean a file that consists mainly of ASCII characters ... and that uses newline characters to give humans the perception of lines

\bigskip
{\footnotesize
Norman Matloff (2011) \\
{\mdlit The Art of R Programming}
}

\end{frame}

%------------------------------------------------

\begin{frame}
\frametitle{Plain Text Files}

\bi
  \item By text files we mean plain text files
  \item Plain text as an umbrella term for any file that is in a human-readable form (\code{.txt, .csv, .xml, .html})
  \item Text files stored as a sequence of characters
  \item Each character stored as a single byte of data
  \item Data is arranged in rows, with several values stored on each row
  \item Text files that can be read and manipulated with a text editor
\ei

\end{frame}

%------------------------------------------------

\begin{frame}
\frametitle{Mandatory Reading}

{\large \textbf{Introduction to Data Technologies (ItDT)} \\
by Paul Murrell

\bigskip
{\mdlit 
\bi
  \item Preface
  \item Chap 1: Introduction
  \item Chap 5: Data Storage
\ei}
}

\end{frame}

%------------------------------------------------

\begin{frame}
\begin{center}
\Huge{\hilit{Tabular Datasets}}
\end{center}
\end{frame}

%------------------------------------------------

\begin{frame}[fragile]
\frametitle{Data Tables}

\begin{center}
{\Large How to store tables in a file?}

\bigskip

 \begin{tabular}{| l | l | l |}
  \hline
name & gender & height \\
  \hline
Leia Skywalker & female & 1.50 \\
  \hline
Luke Skywalker & male & 1.72 \\
  \hline
Han Solo & male & 1.80 \\
  \hline
 \end{tabular}
\end{center}

\end{frame}

%------------------------------------------------

\begin{frame}[fragile]
\frametitle{Storing a Data Table}
\begin{center}
\ig[width=10cm]{images/spreadsheet.pdf}
\end{center}
\end{frame}

%------------------------------------------------

\begin{frame}[fragile]
\frametitle{How NOT to store a Data Table}
\begin{center}
\ig[width=10cm]{images/avoid_excel}
\end{center}
\end{frame}

%------------------------------------------------

\begin{frame}[fragile]
\frametitle{}
\begin{center}
\ig[width=9cm]{images/kitten.pdf}
\end{center}
\end{frame}

%------------------------------------------------

\begin{frame}[fragile]
\frametitle{Dataset ``starwarstoy''}

{\scriptsize
\begin{center}
 \begin{tabular}{| l | l | l | l | l | l | l |}
  \hline
name & gender & height & weight & jedi & species & weapon \\
  \hline
Luke Skywalker & male & 1.72 & 77 & jedi & human & lightsaber \\
  \hline
Leia Skywalker & female & 1.50 & 49 & no\_jedi & human & blaster \\
  \hline
Obi-Wan Kenobi & male & 1.82 & 77 & jedi & human & lightsaber \\
  \hline
Han Solo & male & 1.80 & 80 & no\_jedi & human & blaster \\
  \hline
R2-D2 & male & 0.96 & 32 & no\_jedi & droid & unarmed \\
  \hline
C-3PO & male & 1.67 & 75 & no\_jedi & droid & unarmed \\
  \hline
Yoda & male & 0.66 & 17 & jedi & yoda & lightsaber \\
  \hline
Chewbacca & male & 2.28 & 112 & no\_jedi & wookiee & bowcaster \\
  \hline
 \end{tabular}
\end{center}
}

\tiny Source: Wookiepedia \url{http://starwars.wikia.com/wiki}

\end{frame}

%------------------------------------------------

\begin{frame}
\frametitle{Data Table (computationally)}

\begin{center}
{\Large How to store data cells?}

\bigskip
{\Large What type of format?}
\end{center}

\end{frame}

%------------------------------------------------

\begin{frame}
\frametitle{Character Delimited Text}

\bi
  \item A common way to store data in tabular form is via text files 
  \item To store the data we need a way to separate data values
  \item Each line represents a ``row''
  \item The idea of ``columns'' is conveyed with delimiters 
  \item In summary, fields within each line are separated by the \textbf{delimiter}
  \item Quotation marks are used when the delimiter character occurs within one of the fields
\ei

\end{frame}

%------------------------------------------------

\begin{frame}
\frametitle{Plain Text Formats}

\bbi
  \item There are two main subtypes of plain text format, depending on how the separated values are identified in a row
  \item Delimited formats
  \item Fixed-width formats
\ei

\end{frame}

%------------------------------------------------

\begin{frame}
\frametitle{Delimited Formats}

In a delimited format, values within a row are separated by a special character, or \textbf{delimiter}

\bigskip

{\large
\begin{center}
 \begin{tabular}{l l}
  \hline
  Delimiter & Description \\
  \hline
  \code{" "} & white space \\
  \code{","} & comma \\
  \code{"\textbackslash t"} & tab \\
  \code{";"} & semicolon \\
  \hline
 \end{tabular}
\end{center}
}

\end{frame}

%------------------------------------------------

\begin{frame}[fragile]
\frametitle{Space Delimited (\code{txt})}

{\small
\begin{verbatim}
name gender height weight jedi species weapon
"Luke Skywalker" male 1.72 77 jedi human lightsaber
"Leia Skywalker" female 1.50 49 no_jedi human blaster
"Obi-Wan Kenobi" male 1.82 77 jedi human lightsaber
"Han Solo" male 1.80 80 no_jedi human blaster
"R2-D2" male 0.96 32 no_jedi droid unarmed
"C-3PO" male 1.67 75 no_jedi droid unarmed
"Yoda" male 0.66 17 jedi yoda lightsaber
"Chewbacca" male 2.28 112 no_jedi wookiee bowcaster
\end{verbatim}
}

\end{frame}

%------------------------------------------------

\begin{frame}[fragile]
\frametitle{Comma Delimited (\code{csv})}

{\small
\begin{verbatim}
name,gender,height,weight,jedi,species,weapon
Luke Skywalker,male,1.72,77,jedi,human,lightsaber
Leia Skywalker,female,1.50,49,no_jedi,human,blaster
Obi-Wan Kenobi,male,1.82,77,jedi,human,lightsaber
Han Solo,male,1.80,80,no_jedi,human,blaster
R2-D2,male,0.96,32,no_jedi,droid,unarmed
C-3PO,male,1.67,75,no_jedi,droid,unarmed
Yoda,male,0.66,17,jedi,yoda,lightsaber
Chewbacca,male,2.28,112,no_jedi,wookiee,bowcaster
\end{verbatim}
}

\end{frame}

%------------------------------------------------

\begin{frame}[fragile]
\frametitle{Tab Delimited (\code{txt, tsv})}

{\small
\begin{verbatim}
name  gender  height  weight  jedi  species weapon
"Luke Skywalker"  male  1.72  77  jedi  human lightsaber
"Leia Skywalker"  female  1.50  49  no_jedi human blaster
"Obi-Wan Kenobi"  male  1.82  77  jedi  human lightsaber
"Han Solo"  male  1.80  80  no_jedi human blaster
"R2-D2" male  0.96  32  no_jedi droid unarmed
"C-3PO" male  1.67  75  no_jedi droid unarmed
"Yoda"  male  0.66  17  jedi  yoda  lightsaber
"Chewbacca" male  2.28  112 no_jedi wookiee bowcaster
\end{verbatim}
}

\end{frame}

%------------------------------------------------

\begin{frame}
\frametitle{Fixed-width Formats}

\bbi
  \item In a fixed-width format, each value is allocated a \textbf{fixed number of characters} within every row
\ei

\end{frame}

%------------------------------------------------

\begin{frame}[fragile]
\frametitle{Fixed-Width (\code{txt})}

{\small
\begin{verbatim}
name              gender  height weight jedi  
"Luke Skywalker"  male    1.72   77     jedi   
"Leia Skywalker"  female  1.50   49     no_jedi
"Obi-Wan Kenobi"  male    1.82   77     jedi  
"Han Solo"        male    1.80   80     no_jedi
"R2-D2"           male    0.96   32     no_jedi
"C-3PO"           male    1.67   75     no_jedi
"Yoda"            male    0.66   17     jedi
"Chewbacca"       male    2.28   112    no_jedi
\end{verbatim}
}

\end{frame}

%------------------------------------------------

\begin{frame}
\frametitle{In Summary}

\bb{Plain Text Formats}
\bbi
  \item The simplest way to store information in computer memory is a file with a \textbf{plain text format}
  \item The basic conceptual structure of a plain text format is that the \textbf{data are arranged in rows}, with several values stored on each row
  \item The main characteristic of a plain text format is that all of the information in a file, even numeric information, is stored as text
\ei
\eb

\end{frame}

%------------------------------------------------

\begin{frame}
\begin{center}
\Huge{\hilit{Importing Data Tables in R}}
\end{center}
\end{frame}

%------------------------------------------------

\begin{frame}
\frametitle{R Data Import Manual}

There's a wide range of ways and options to import data tables in R.

\bigskip
The authoritative document to know almost all about importing (and exporting) data is the manual \textbf{R Data Import/Export} \\
{\footnotesize \url{http://cran.r-project.org/doc/manuals/r-release/R-data.html}}

\end{frame}

%------------------------------------------------

\begin{frame}
\frametitle{Importing Data Tables}

The most common way to read and import tables in R is by using {\hilit \code{read.table()}} and friends

\bigskip
The read data output is always a \textbf{data.frame}

\end{frame}

%------------------------------------------------

\begin{frame}[fragile]
\frametitle{\code{read.table()}}

{\footnotesize
\begin{verbatim}
read.table(file, header = FALSE, sep = "", quote = "\"'",
           dec = ".", row.names, col.names,
           as.is = !stringsAsFactors,
           na.strings = "NA", colClasses = NA, nrows = -1,
           skip = 0, check.names = TRUE, 
           fill = !blank.lines.skip,
           strip.white = FALSE, blank.lines.skip = TRUE,
           comment.char = "#",
           allowEscapes = FALSE, flush = FALSE,
           stringsAsFactors = default.stringsAsFactors(),
           fileEncoding = "", encoding = "unknown", text, 
           skipNul = FALSE)
\end{verbatim}
}

\end{frame}

%------------------------------------------------

\begin{frame}
\frametitle{Some \code{read.table() arguments}}
{\small
\begin{center}
 \begin{tabular}{l l}
  \hline
   Argument & Description \\
  \hline
  \code{file} & name of file \\  
  \code{header} & whether column names are in 1st line \\
  \code{sep} & field separator \\  
  \code{quote} & quoting characters \\
  \code{dec} & character for decimal point \\
  \code{row.names} & optional vector of row names \\
  \code{col.names} & optional vector of column names \\
  \code{na.strings} & character treated as missing values \\
  \code{colClasses} & optional vector of classes for columns \\
  \code{nrows} & maximum number of rows to read in \\
  \code{skip} & number of lines to skip before reading data \\
  \code{check.names} & check valid column names \\
  \code{stringsAsFactors} & should characters be converted to factors \\
  \hline
 \end{tabular}
\end{center}
}
\end{frame}

%------------------------------------------------

\begin{frame}[fragile]
\frametitle{Consider some dataset}
\begin{center}
\ig[width=11cm]{images/read_df1.pdf}
\end{center}
\end{frame}

%------------------------------------------------

\begin{frame}[fragile]
\frametitle{Arguments for \code{read.table()}}
\begin{center}
\ig[width=11cm]{images/read_df2.pdf}
\end{center}
\end{frame}

%------------------------------------------------

\begin{frame}
\frametitle{Assumption}

For simplicity's sake, we'll assume that all data files are located in your working directory:

\bigskip
e.g. \code{"/Users/Gaston/Documents"}

\end{frame}

%------------------------------------------------

\begin{frame}[fragile]
\frametitle{\code{starwarstoy.txt}}

{\small
\begin{verbatim}
name gender height weight jedi species weapon
"Luke Skywalker" male 1.72 77 jedi human lightsaber
"Leia Skywalker" female 1.5 49 no_jedi human blaster
"Obi-Wan Kenobi" male 1.82 77 jedi human lightsaber
"Han Solo" male 1.8 80 no_jedi human blaster
"R2-D2" male 0.96 32 no_jedi droid unarmed
"C-3PO" male 1.67 75 no_jedi droid unarmed
"Yoda" male 0.66 17 jedi yoda lightsaber
"Chewbacca" male 2.28 112 no_jedi wookiee bowcaster
\end{verbatim}
}

{\scriptsize Lecture data files at: \\
\url{https://github.com/gastonstat/stat133/tree/master/datasets}}
\end{frame}

%------------------------------------------------

\begin{frame}[fragile]
\frametitle{Reading \code{starwarstoy.txt}}

Blank space delimiter \code{" "}

\begin{knitrout}\footnotesize
\definecolor{shadecolor}{rgb}{0.969, 0.969, 0.969}\color{fgcolor}\begin{kframe}
\begin{alltt}
\hlcom{# using read.table()}
\hlstd{sw_txt} \hlkwb{<-} \hlkwd{read.table}\hlstd{(}
  \hlkwc{file} \hlstd{=} \hlstr{"starwarstoy.txt"}\hlstd{,}
  \hlkwc{header} \hlstd{=} \hlnum{TRUE}\hlstd{)}
\end{alltt}
\end{kframe}
\end{knitrout}

Note: by default \code{read.table()} (and friends) convert character strings into factors

\end{frame}

%------------------------------------------------

\begin{frame}[fragile]
\frametitle{Reading \code{starwarstoy.txt}}

Compare to this other option:

\begin{knitrout}\footnotesize
\definecolor{shadecolor}{rgb}{0.969, 0.969, 0.969}\color{fgcolor}\begin{kframe}
\begin{alltt}
\hlcom{# first column as row names}
\hlstd{sw_txt1} \hlkwb{<-} \hlkwd{read.table}\hlstd{(}
  \hlkwc{file} \hlstd{=} \hlstr{"starwarstoy.txt"}\hlstd{,}
  \hlkwc{header} \hlstd{=} \hlnum{TRUE}\hlstd{,}
  \hlkwc{row.names} \hlstd{=} \hlnum{1}\hlstd{)}
\end{alltt}
\end{kframe}
\end{knitrout}

\end{frame}

%------------------------------------------------

\begin{frame}[fragile]
\frametitle{Reading \code{starwarstoy.txt}}

Limit the number of rows to read in (first 4 individuals):

\begin{knitrout}\footnotesize
\definecolor{shadecolor}{rgb}{0.969, 0.969, 0.969}\color{fgcolor}\begin{kframe}
\begin{alltt}
\hlcom{# first column as row names}
\hlstd{sw_txt2} \hlkwb{<-} \hlkwd{read.table}\hlstd{(}
  \hlkwc{file} \hlstd{=} \hlstr{"starwarstoy.txt"}\hlstd{,}
  \hlkwc{header} \hlstd{=} \hlnum{TRUE}\hlstd{,}
  \hlkwc{row.names} \hlstd{=} \hlnum{1}\hlstd{,}
  \hlkwc{nrows} \hlstd{=} \hlnum{4}\hlstd{)}
\end{alltt}
\end{kframe}
\end{knitrout}

\end{frame}

%------------------------------------------------

\begin{frame}[fragile]
\frametitle{Reading \code{starwarstoy.txt}}

Let's skip the first row (no header):

\begin{knitrout}\footnotesize
\definecolor{shadecolor}{rgb}{0.969, 0.969, 0.969}\color{fgcolor}\begin{kframe}
\begin{alltt}
\hlcom{# first column as row names}
\hlstd{sw_txt3} \hlkwb{<-} \hlkwd{read.table}\hlstd{(}
  \hlkwc{file} \hlstd{=} \hlstr{"starwarstoy.txt"}\hlstd{,}
  \hlkwc{header} \hlstd{=} \hlnum{FALSE}\hlstd{,}
  \hlkwc{skip} \hlstd{=} \hlnum{1}\hlstd{,}
  \hlkwc{row.names} \hlstd{=} \hlnum{1}\hlstd{,}
  \hlkwc{nrows} \hlstd{=} \hlnum{4}\hlstd{)}
\end{alltt}
\end{kframe}
\end{knitrout}

\end{frame}

%------------------------------------------------

\begin{frame}[fragile]
\frametitle{\code{starwarstoy.csv}}

{\small
\begin{verbatim}
name,gender,height,weight,jedi,species,weapon
Luke Skywalker,male,1.72,77,jedi,human,lightsaber
Leia Skywalker,female,1.5,49,no_jedi,human,blaster
Obi-Wan Kenobi,male,1.82,77,jedi,human,lightsaber
Han Solo,male,1.8,80,no_jedi,human,blaster
R2-D2,male,0.96,32,no_jedi,droid,unarmed
C-3PO,male,1.67,75,no_jedi,droid,unarmed
Yoda,male,0.66,17,jedi,yoda,lightsaber
Chewbacca,male,2.28,112,no_jedi,wookiee,bowcaster
\end{verbatim}
}

\end{frame}

%------------------------------------------------

\begin{frame}[fragile]
\frametitle{Reading \code{starwarstoy.csv}}

Comma delimiter \code{","}

\begin{knitrout}\footnotesize
\definecolor{shadecolor}{rgb}{0.969, 0.969, 0.969}\color{fgcolor}\begin{kframe}
\begin{alltt}
\hlcom{# using read.table()}
\hlstd{sw_csv} \hlkwb{<-} \hlkwd{read.table}\hlstd{(}\hlkwc{file} \hlstd{=} \hlstr{"starwarstoy.csv"}\hlstd{,}
                     \hlkwc{header} \hlstd{=} \hlnum{TRUE}\hlstd{,}
                     \hlkwc{sep} \hlstd{=} \hlstr{","}\hlstd{)}

\hlcom{# using read.csv()}
\hlstd{sw_csv} \hlkwb{<-} \hlkwd{read.csv}\hlstd{(}\hlkwc{file} \hlstd{=} \hlstr{"starwarstoy.csv"}\hlstd{)}
\end{alltt}
\end{kframe}
\end{knitrout}

\end{frame}

%------------------------------------------------

\begin{frame}[fragile]
\frametitle{\code{starwarstoy.csv2}}

{\small
\begin{verbatim}
name;gender;height;weight;jedi;species;weapon
Luke Skywalker;male;1,72;77;jedi;human;lightsaber
Leia Skywalker;female;1,5;49;no_jedi;human;blaster
Obi-Wan Kenobi;male;1,82;77;jedi;human;lightsaber
Han Solo;male;1,8;80;no_jedi;human;blaster
R2-D2;male;0,96;32;no_jedi;droid;unarmed
C-3PO;male;1,67;75;no_jedi;droid;unarmed
Yoda;male;0,66;17;jedi;yoda;lightsaber
Chewbacca;male;2,28;112;no_jedi;wookiee;bowcaster
\end{verbatim}
}

\end{frame}

%------------------------------------------------

\begin{frame}[fragile]
\frametitle{Reading \code{starwarstoy.csv2}}

Semicolon delimiter \code{","} and decimal symbol \code{","}

\begin{knitrout}\footnotesize
\definecolor{shadecolor}{rgb}{0.969, 0.969, 0.969}\color{fgcolor}\begin{kframe}
\begin{alltt}
\hlcom{# using read.table()}
\hlstd{sw_csv2} \hlkwb{<-} \hlkwd{read.table}\hlstd{(}\hlkwc{file} \hlstd{=} \hlstr{"starwarstoy.csv"}\hlstd{,}
                      \hlkwc{header} \hlstd{=} \hlnum{TRUE}\hlstd{,}
                      \hlkwc{sep} \hlstd{=} \hlstr{";"}\hlstd{,} \hlkwc{dec} \hlstd{=} \hlstr{","}\hlstd{)}

\hlcom{# using read.csv2()}
\hlstd{sw_csv2} \hlkwb{<-} \hlkwd{read.csv2}\hlstd{(}\hlkwc{file} \hlstd{=} \hlstr{"starwarstoy.csv2"}\hlstd{)}
\end{alltt}
\end{kframe}
\end{knitrout}

\end{frame}

%------------------------------------------------

\begin{frame}[fragile]
\frametitle{\code{starwarstoy.tsv}}

{\small
\begin{verbatim}
name  gender	height	weight	jedi	species	weapon
Luke Skywalker	male	1.72	77	jedi	human	lightsaber
Leia Skywalker	female	1.5	49	no_jedi	human	blaster
Obi-Wan Kenobi	male	1.82	77	jedi	human	lightsaber
Han Solo	male	1.8	80	no_jedi	human	blaster
R2-D2	male	0.96	32	no_jedi	droid	unarmed
C-3PO	male	1.67	75	no_jedi	droid	unarmed
Yoda	male	0.66	17	jedi	yoda	lightsaber
Chewbacca	male	2.28	112	no_jedi	wookiee	bowcaster
\end{verbatim}
}

\end{frame}

%------------------------------------------------

\begin{frame}[fragile]
\frametitle{Reading \code{starwarstoy.tsv}}

Tab delimiter \code{"\textbackslash t"}

\begin{knitrout}\footnotesize
\definecolor{shadecolor}{rgb}{0.969, 0.969, 0.969}\color{fgcolor}\begin{kframe}
\begin{alltt}
\hlcom{# using read.table()}
\hlstd{sw_tsv} \hlkwb{<-} \hlkwd{read.table}\hlstd{(}\hlkwc{file} \hlstd{=} \hlstr{"starwarstoy.tsv"}\hlstd{,}
                     \hlkwc{header} \hlstd{=} \hlnum{TRUE}\hlstd{,}
                     \hlkwc{sep} \hlstd{=} \hlstr{"\textbackslash{}t"}\hlstd{)}

\hlcom{# using read.delim()}
\hlstd{sw_tsv} \hlkwb{<-} \hlkwd{read.delim}\hlstd{(}\hlkwc{file} \hlstd{=} \hlstr{"starwarstoy.tsv"}\hlstd{)}
\end{alltt}
\end{kframe}
\end{knitrout}

\end{frame}

%------------------------------------------------

\begin{frame}[fragile]
\frametitle{\code{starwarstoy.dat}}

{\small
\begin{verbatim}
name%gender%height%weight%jedi%species%weapon
Luke Skywalker%male%1.72%77%jedi%human%lightsaber
Leia Skywalker%female%1.5%49%no_jedi%human%blaster
Obi-Wan Kenobi%male%1.82%77%jedi%human%lightsaber
Han Solo%male%1.8%80%no_jedi%human%blaster
R2-D2%male%0.96%32%no_jedi%droid%unarmed
C-3PO%male%1.67%75%no_jedi%droid%unarmed
Yoda%male%0.66%17%jedi%yoda%lightsaber
Chewbacca%male%2.28%112%no_jedi%wookiee%bowcaster
\end{verbatim}
}

\end{frame}

%------------------------------------------------

\begin{frame}[fragile]
\frametitle{Reading \code{starwarstoy.dat}}

Note that this file has \code{"\%"} as delimiter
\begin{knitrout}\footnotesize
\definecolor{shadecolor}{rgb}{0.969, 0.969, 0.969}\color{fgcolor}\begin{kframe}
\begin{alltt}
\hlcom{# using read.table()}
\hlstd{sw_dat} \hlkwb{<-} \hlkwd{read.table}\hlstd{(}\hlkwc{file} \hlstd{=} \hlstr{"starwarstoy.dat"}\hlstd{,}
                     \hlkwc{header} \hlstd{=} \hlnum{TRUE}\hlstd{,}
                     \hlkwc{sep} \hlstd{=} \hlstr{"%"}\hlstd{)}
\end{alltt}
\end{kframe}
\end{knitrout}

\end{frame}

%------------------------------------------------

\begin{frame}
\frametitle{\code{read.table()} and friends}

\begin{center}
 \begin{tabular}{l l}
  \hline
   Function & Description \\
  \hline
  \code{read.csv()} & comma separated values \\  
  \code{read.csv2()} & semicolon separated values (Europe) \\
  \code{read.delim()} & tab separated values \\  
  \code{read.delim2()} & tab separated values (Europe) \\
   \hline
 \end{tabular}
\end{center}

There is also the \code{read.fwf()} function for reading a table of \textbf{fixed width format}

\end{frame}

%------------------------------------------------

\begin{frame}
\frametitle{Considerations}

\bb{What is the field separator?}
\bi
  \item space \code{" "}
  \item tab \code{"\textbackslash t"}
  \item comman \code{","}
  \item semicolon \code{";"}
  \item other?
\ei
\eb

\end{frame}

%------------------------------------------------

\begin{frame}
\frametitle{Considerations}

\bb{Does the data file contains:}
\bi
  \item row names?
  \item column names?
  \item missing values?
  \item special characters?
\ei
\eb

\end{frame}

%------------------------------------------------

\begin{frame}
\frametitle{Summary}

\bb{So far ...}
\bi
  \item There are multiple ways to import data tables
  \item The workhorse function is \code{read.table()}
  \item But you can use the other wrappers, e.g. \code{read.csv()}
  \item The output is a \code{"data.frame"} object
\ei
\eb

\end{frame}

%------------------------------------------------

\begin{frame}
\frametitle{Location of data file}

Sometimes the issue is not the type of file but its location
\bbi
  \item zip file
  \item url (\code{http} standard)
  \item url (\code{https} HTTP secure)
\ei

\end{frame}

%------------------------------------------------

\begin{frame}
\frametitle{Reading compressed files}

R provides various \code{connections} functions for opening and reading compressed files:
\bi
  \item \code{unz()} reads only a single zip file
  \item \code{gzfile()} for gzip, bzip2, xz, lzma
  \item \code{bzfile()} for bzip2
  \item \code{xzfile()} for xz
\ei

You pass a connection to the argument \code{file} in any of the reading files functions.

\end{frame}

%------------------------------------------------

\begin{frame}
\frametitle{Reading zip files}

\code{unz(description, filename)}

\bi
  \item \code{description} is the full path to the zip file
  with \code{.zip} extension if required
  \item \code{filename} is the name of the file
\ei

\end{frame}

%------------------------------------------------

\begin{frame}[fragile]
\frametitle{Reading a single zip file}

\code{starwarstoy.zip} contains a copy of the file \code{starwarstoy.txt}; to import it in R type:
\begin{knitrout}\footnotesize
\definecolor{shadecolor}{rgb}{0.969, 0.969, 0.969}\color{fgcolor}\begin{kframe}
\begin{alltt}
\hlstd{sw_zip} \hlkwb{<-} \hlkwd{read.table}\hlstd{(}
  \hlkwc{file} \hlstd{=} \hlkwd{unz}\hlstd{(}\hlkwc{description} \hlstd{=} \hlstr{"starwarstoy.zip"}\hlstd{,}
             \hlstr{"starwarstoy.txt"}\hlstd{)}
\hlstd{)}
\end{alltt}
\end{kframe}
\end{knitrout}

\end{frame}

%------------------------------------------------

\begin{frame}[fragile]
\frametitle{Connection for the web}

\bb{Using \code{url()}}
 \begin{verbatim}
url(description, open = "", blocking = TRUE,
    encoding = getOption("encoding"))
 \end{verbatim}
\eb

The main input for \code{url()} is the \code{description} which has to be a complete URL, including scheme such as \code{http://}, \code{ftp://}, or \code{file://}

\end{frame}

%------------------------------------------------

\begin{frame}[fragile]
\frametitle{Example of \code{url} connection}

For instance, let's create an url connection to 
\begin{knitrout}\tiny
\definecolor{shadecolor}{rgb}{0.969, 0.969, 0.969}\color{fgcolor}\begin{kframe}
\begin{alltt}
\hlcom{# creating a url connection to some file}
\hlstd{edu} \hlkwb{<-} \hlkwd{url}\hlstd{(}\hlstr{"http://gastonsanchez.com/education.csv"}\hlstd{)}

\hlcom{# what's in 'edu'}
\hlstd{edu}
\end{alltt}
\begin{verbatim}
##                              description 
## "http://gastonsanchez.com/education.csv" 
##                                    class 
##                                    "url" 
##                                     mode 
##                                      "r" 
##                                     text 
##                                   "text" 
##                                   opened 
##                                 "closed" 
##                                 can read 
##                                    "yes" 
##                                can write 
##                                     "no"
\end{verbatim}
\begin{alltt}
\hlcom{# is open?}
\hlkwd{isOpen}\hlstd{(edu)}
\end{alltt}
\begin{verbatim}
## [1] FALSE
\end{verbatim}
\end{kframe}
\end{knitrout}

\end{frame}

%------------------------------------------------

\begin{frame}
\frametitle{About Connections}

\bb{Should we care?}
\bi
  \item Most of the times we don't need to explicitly use \code{url()}. 
  \item Connections can be used anywhere a file name could be passed to functions like \code{read.table()} 
  \item Usually, the reading functions ---eg \code{read.table()}, \code{read.csv()}--- will take care of the URL connection for us.
  \item However, there may be occassions in which we will need to specify a \code{url()} connection.
\ei
\eb

\end{frame}

%------------------------------------------------

\begin{frame}
\frametitle{Good to Know}

\bb{Terms of Service}
Some times, reading data directly from a website may be against the \hilit{terms of use} of the site.
\eb

\bb{Web Politeness}
When you're reading (and ``playing'' with) content from a web page, make a local copy as a courtesy to the owner of the web site so you don't overload their server by constantly rereading the page. To make a copy from inside of R, look at the \code{download.file()} function. 
\eb

\end{frame}

%------------------------------------------------

\begin{frame}[fragile]
\frametitle{Downloading Files}

\bb{Downloading files from the web}
It is good advice to download a copy of the file to your computer, and then play with it. 

\bigskip

Let's use \code{download.file()} to save a copy in our working directory. In this case we create the file \code{education.csv}

\begin{knitrout}\footnotesize
\definecolor{shadecolor}{rgb}{0.969, 0.969, 0.969}\color{fgcolor}\begin{kframe}
\begin{alltt}
\hlcom{# download a copy in your working directory}
\hlkwd{download.file}\hlstd{(}\hlstr{"http://gastonsanchez.com/education.csv"}\hlstd{,}
              \hlstr{"education.csv"}\hlstd{)}
\end{alltt}
\end{kframe}
\end{knitrout}
\eb

\end{frame}

%------------------------------------------------

\begin{frame}[fragile]
\frametitle{Reading files via \code{https}}

To read data tables via \code{https} (to connect via a secured HTTP) we need to use the R package \code{"RCurl"}
\begin{knitrout}\footnotesize
\definecolor{shadecolor}{rgb}{0.969, 0.969, 0.969}\color{fgcolor}\begin{kframe}
\begin{alltt}
\hlcom{# load package RCurl}
\hlkwd{library}\hlstd{(RCurl)}

\hlcom{# URL of data file}
\hlstd{url} \hlkwb{<-} \hlkwd{getURL}\hlstd{(}\hlstr{"https://???"}\hlstd{)}

\hlcom{# import data in R (through a text connection)}
\hlstd{df} \hlkwb{<-} \hlkwd{read.csv}\hlstd{(}\hlkwd{textConnection}\hlstd{(url),}
               \hlkwc{row.names} \hlstd{=} \hlnum{1}\hlstd{,} \hlkwc{header} \hlstd{=} \hlnum{TRUE}\hlstd{)}
\end{alltt}
\end{kframe}
\end{knitrout}

\end{frame}

%------------------------------------------------

\begin{frame}
\frametitle{Clicker poll}

Which of the following sentences is TRUE \\
\bi
  \item[A)] spreadsheet formats have no limits on the numbers of columns and rows
  \item[B)] spreadsheet format is always better than a plain text or binary data format
  \item[C)] a lot of unnecessary additional information is stored in a spreadsheet file
  \item[D)] All of the above
\ei

\end{frame}

%------------------------------------------------

\begin{frame}
\begin{center}
\Huge{\hilit{R package \code{"readr"}}}
\end{center}
\end{frame}

%------------------------------------------------

\begin{frame}
\frametitle{Package \code{"readr"}}

The package \code{"readr"} (by Wickham \textit{et al}) is a new package that makes it easy to read many types of tabular data

\bigskip

{\scriptsize
\url{http://blog.rstudio.org/2015/04/09/readr-0-1-0/}

\url{http://cran.r-project.org/web/packages/readr/vignettes/design.html}
}
\end{frame}

%------------------------------------------------

\begin{frame}[fragile]
\frametitle{Package \code{"readr"}}
\begin{knitrout}\footnotesize
\definecolor{shadecolor}{rgb}{0.969, 0.969, 0.969}\color{fgcolor}\begin{kframe}
\begin{alltt}
\hlcom{# remember to install 'readr'}
\hlkwd{install.packages}\hlstd{(}\hlstr{"readr"}\hlstd{)}

\hlcom{# load it}
\hlkwd{library}\hlstd{(readr)}
\end{alltt}
\end{kframe}
\end{knitrout}
\end{frame}

%------------------------------------------------

\begin{frame}
\frametitle{\code{"readr"} Functions}

\bi
  \item Fixed width files with \code{read\_table()} and \code{read\_fwf()}
  \item Delimited files with \code{read\_delim()}, \code{read\_csv()}, \code{read\_tsv()}, and \code{read\_csv2()}
\ei

\end{frame}

%------------------------------------------------

\begin{frame}
\frametitle{About \code{"readr"}}

\bb{\code{"readr"} functions ...}
\bbi
  \item are around 10x faster than base functions
  \item are more consistent (better designed)
  \item produce data frames that are easier to use
  \item they have more flexible column specification
\ei
\eb

\end{frame}

%------------------------------------------------

\begin{frame}
\frametitle{Input Arguments}

\bbi
  \item \code{file}
  \item \code{col\_names}
  \item \code{col\_types}
  \item \code{progress}
\ei

\end{frame}

%------------------------------------------------

\begin{frame}
\frametitle{Input Arguments}

{\hilit \code{file}} gives the file to read; a url or local path. A local path can point to a a zipped, bzipped, xzipped, or gzipped file – it'll be automatically uncompressed in memory before reading. 

\end{frame}

%------------------------------------------------

\begin{frame}
\frametitle{Input Arguments}

{\hilit \code{col\_names}}: describes the column names (equivalent to \code{header} in base R). It has three possible values:

\bi
  \item \code{TRUE} will use the the first row of data as column names.
  \item \code{FALSE} will number the columns sequentially.
  \item A character vector to use as column names.
\ei

\end{frame}

%------------------------------------------------

\begin{frame}
\frametitle{Input Arguments}

{\hilit \code{col\_types}} (equivalent to \code{colClasses} automatically detects column types:

\bi
  \item \code{col\_logical()} contains only logical values
  \item \code{col\_integer()} integers
  \item \code{col\_double())} doubles (reals)
  \item \code{col\_euro\_double()} ``Euro'' doubles that use commas \code{","} as decimal separator
  \item \code{col\_date()} Y-m-d dates
  \item \code{col\_datetime()}: ISO8601 date times
  \item \code{col\_character()}: everything else
\ei

\end{frame}

%------------------------------------------------

\begin{frame}
\frametitle{Column Types Correspondence}

\begin{center}
 \begin{tabular}{l c}
  \hline
   Type & Abbreviation \\
  \hline
  \code{col\_logical()} & \code{l} \\  
  \code{col\_integer()} & \code{i} \\
  \code{col\_numeric()} & \code{n} \\  
  \code{col\_double()} & \code{d} \\  
  \code{col\_euro\_double()} & \code{e} \\
  \code{col\_date()} & \code{D}  \\
  \code{col\_datetime()} & \code{T} \\
  \code{col\_character()} & \code{c} \\
  \code{col\_skip()} & \code{\_} \\
  \hline
 \end{tabular}
\end{center}

\end{frame}

%------------------------------------------------

\begin{frame}
\frametitle{Column Types}

\bb{Overriding default choice of \code{col\_types}}
Use a compact string: {\hilit \code{"dc\_\_d"}}. Each letter corresponds to a column so this specification means: read first column as double, second as character, skip the next two and read the last column as a double. (There's no way to use this form with column types that need parameters.)
\eb

\end{frame}

%------------------------------------------------

\begin{frame}[fragile]
\frametitle{Column Types}

\bb{Overriding default choice of \code{col\_types}}
Another way to override the default choices of column types is by passing a list of \code{col\_...} objects:
\eb
\begin{knitrout}\footnotesize
\definecolor{shadecolor}{rgb}{0.969, 0.969, 0.969}\color{fgcolor}\begin{kframe}
\begin{alltt}
\hlkwd{read_csv}\hlstd{(}\hlstr{"iris.csv"}\hlstd{,} \hlkwc{col_types} \hlstd{=} \hlkwd{list}\hlstd{(}
  \hlkwc{Sepal.Length} \hlstd{=} \hlkwd{col_double}\hlstd{(),}
  \hlkwc{Sepal.Width} \hlstd{=} \hlkwd{col_double}\hlstd{(),}
  \hlkwc{Petal.Length} \hlstd{=} \hlkwd{col_double}\hlstd{(),}
  \hlkwc{Petal.Width} \hlstd{=} \hlkwd{col_double}\hlstd{(),}
  \hlkwc{Species} \hlstd{=} \hlkwd{col_factor}\hlstd{(}\hlkwd{c}\hlstd{(}\hlstr{"setosa"}\hlstd{,} \hlstr{"versicolor"}\hlstd{,} \hlstr{"virginica"}\hlstd{))}
\hlstd{))}
\end{alltt}
\end{kframe}
\end{knitrout}

\end{frame}

%------------------------------------------------

\begin{frame}[fragile]
\frametitle{Output}

\bi
  \item Characters are never automatically converted to factors
  \item Column names are left as is \\
  (i.e. there is no \code{check.names = TRUE})
  \item Use backticks to refer to variables with unusual names:
\begin{knitrout}\footnotesize
\definecolor{shadecolor}{rgb}{0.969, 0.969, 0.969}\color{fgcolor}\begin{kframe}
\begin{alltt}
\hlstd{df}\hlopt{$}\hlstd{`Income ($000)`}
\end{alltt}
\end{kframe}
\end{knitrout}
  \item Row names are never set
  \item The output has class \\ 
\begin{knitrout}\footnotesize
\definecolor{shadecolor}{rgb}{0.969, 0.969, 0.969}\color{fgcolor}\begin{kframe}
\begin{alltt}
\hlkwd{c}\hlstd{(}\hlstr{"tbl_df"}\hlstd{,} \hlstr{"tbl"}\hlstd{,} \hlstr{"data.frame"}\hlstd{)}
\end{alltt}
\end{kframe}
\end{knitrout}
\ei


\end{frame}

%------------------------------------------------

\begin{frame}[fragile]
\frametitle{\code{"starwarstoy.csv"}}

{\small
\begin{verbatim}
name,gender,height,weight,jedi,species,weapon
Luke Skywalker,male,1.72,77,jedi,human,lightsaber
Leia Skywalker,female,1.50,49,no_jedi,human,blaster
Obi-Wan Kenobi,male,1.82,77,jedi,human,lightsaber
Han Solo,male,1.80,80,no_jedi,human,blaster
R2-D2,male,0.96,32,no_jedi,droid,unarmed
C-3PO,male,1.67,75,no_jedi,droid,unarmed
Yoda,male,0.66,17,jedi,yoda,lightsaber
Chewbacca,male,2.28,112,no_jedi,wookiee,bowcaster
\end{verbatim}
}

\end{frame}

%------------------------------------------------

\begin{frame}[fragile]
\frametitle{String Columns as factors}

By default, functions in \code{"readr"} do not convert character strings into factors. But you can specify what columns to be imported as factors (you must specify the levels):
\begin{knitrout}\footnotesize
\definecolor{shadecolor}{rgb}{0.969, 0.969, 0.969}\color{fgcolor}\begin{kframe}
\begin{alltt}
\hlstd{sw1} \hlkwb{<-} \hlkwd{read_csv}\hlstd{(}
  \hlkwc{file} \hlstd{=} \hlstr{"starwarstoy.csv"}\hlstd{,}
  \hlkwc{col_types} \hlstd{=} \hlkwd{list}\hlstd{(}
    \hlkwc{gender} \hlstd{=} \hlkwd{col_factor}\hlstd{(}\hlkwd{c}\hlstd{(}\hlstr{"male"}\hlstd{,} \hlstr{"female"}\hlstd{)))}
\hlstd{)}
\end{alltt}
\end{kframe}
\end{knitrout}

\end{frame}

%------------------------------------------------

\begin{frame}[fragile]
\frametitle{Importing selected columns}

\code{"readr"} allows you to import specific columns of a dataset
\begin{knitrout}\footnotesize
\definecolor{shadecolor}{rgb}{0.969, 0.969, 0.969}\color{fgcolor}\begin{kframe}
\begin{alltt}
\hlcom{# importing just first 4 columns}
\hlstd{sw4} \hlkwb{<-} \hlkwd{read_csv}\hlstd{(}
  \hlkwc{file} \hlstd{=} \hlstr{"starwarstoy.csv"}\hlstd{,}
  \hlkwc{col_types} \hlstd{=} \hlstr{"ccnn___"}
\hlstd{)}
\end{alltt}
\end{kframe}
\end{knitrout}

\end{frame}

%------------------------------------------------

\begin{frame}
\frametitle{Main functions in \code{"readr"}}

\bi
  \item \code{read\_table()}
  \item \code{read\_delim()}
  \item \code{read\_csv()}
  \item \code{read\_csv2()}
  \item \code{read\_tsv()}
  \item \code{read\_fwf()}
\ei

\end{frame}

%------------------------------------------------

\begin{frame}
\begin{center}
\Huge{\hilit{Foreign Files}}
\end{center}
\end{frame}

%------------------------------------------------

\begin{frame}
\frametitle{Data Table (foreign files)}

\begin{center}
{\Large It is not uncommon to have tabular datasets in foreign files (e.g. from other programs)}
\end{center}

\end{frame}

%------------------------------------------------

\begin{frame}
\frametitle{Files from other programs}

\begin{center}
 \begin{tabular}{l l l}
  \hline
   Type & Package & Function \\
  \hline
  Excel & \code{"gdata"} & \code{read.xls()} \\  
  Excel & \code{"xlsx"} & \code{read.xlsx()} \\
  Excel & \code{"readxl"} & \code{read\_excel()} \\  
  SPSS & \code{"foreign"} & \code{read.spss()} \\
  SAS & \code{"foreign"} & \code{read.ssd()} \\
  SAS & \code{"foreign"} & \code{read.xport()} \\
  Matlab & \code{"R.matlab"} & \code{readMat()} \\
  Stata & \code{"foreign"} & \code{read.dta()} \\
  Octave & \code{"foreign"} & \code{read.octave()} \\
  Minitab & \code{"foreign"} & \code{read.mtp()} \\
  Systat & \code{"foreign"} & \code{read.systat()} \\
  \hline
 \end{tabular}
\end{center}

\end{frame}


\end{document}
