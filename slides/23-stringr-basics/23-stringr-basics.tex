\documentclass[12pt]{beamer}\usepackage[]{graphicx}\usepackage[]{color}
%% maxwidth is the original width if it is less than linewidth
%% otherwise use linewidth (to make sure the graphics do not exceed the margin)
\makeatletter
\def\maxwidth{ %
  \ifdim\Gin@nat@width>\linewidth
    \linewidth
  \else
    \Gin@nat@width
  \fi
}
\makeatother

\definecolor{fgcolor}{rgb}{0.345, 0.345, 0.345}
\newcommand{\hlnum}[1]{\textcolor[rgb]{0.686,0.059,0.569}{#1}}%
\newcommand{\hlstr}[1]{\textcolor[rgb]{0.192,0.494,0.8}{#1}}%
\newcommand{\hlcom}[1]{\textcolor[rgb]{0.678,0.584,0.686}{\textit{#1}}}%
\newcommand{\hlopt}[1]{\textcolor[rgb]{0,0,0}{#1}}%
\newcommand{\hlstd}[1]{\textcolor[rgb]{0.345,0.345,0.345}{#1}}%
\newcommand{\hlkwa}[1]{\textcolor[rgb]{0.161,0.373,0.58}{\textbf{#1}}}%
\newcommand{\hlkwb}[1]{\textcolor[rgb]{0.69,0.353,0.396}{#1}}%
\newcommand{\hlkwc}[1]{\textcolor[rgb]{0.333,0.667,0.333}{#1}}%
\newcommand{\hlkwd}[1]{\textcolor[rgb]{0.737,0.353,0.396}{\textbf{#1}}}%

\usepackage{framed}
\makeatletter
\newenvironment{kframe}{%
 \def\at@end@of@kframe{}%
 \ifinner\ifhmode%
  \def\at@end@of@kframe{\end{minipage}}%
  \begin{minipage}{\columnwidth}%
 \fi\fi%
 \def\FrameCommand##1{\hskip\@totalleftmargin \hskip-\fboxsep
 \colorbox{shadecolor}{##1}\hskip-\fboxsep
     % There is no \\@totalrightmargin, so:
     \hskip-\linewidth \hskip-\@totalleftmargin \hskip\columnwidth}%
 \MakeFramed {\advance\hsize-\width
   \@totalleftmargin\z@ \linewidth\hsize
   \@setminipage}}%
 {\par\unskip\endMakeFramed%
 \at@end@of@kframe}
\makeatother

\definecolor{shadecolor}{rgb}{.97, .97, .97}
\definecolor{messagecolor}{rgb}{0, 0, 0}
\definecolor{warningcolor}{rgb}{1, 0, 1}
\definecolor{errorcolor}{rgb}{1, 0, 0}
\newenvironment{knitrout}{}{} % an empty environment to be redefined in TeX

\usepackage{alltt}
\usepackage{graphicx}
\usepackage{tikz}
\setbeameroption{hide notes}
\setbeamertemplate{note page}[plain]
\usepackage{listings}

% get rid of junk
\usetheme{default}
\usefonttheme[onlymath]{serif}
\beamertemplatenavigationsymbolsempty
\hypersetup{pdfpagemode=UseNone} % don't show bookmarks on initial view

% named colors
\definecolor{offwhite}{RGB}{255,250,240}
\definecolor{gray}{RGB}{155,155,155}

\ifx\notescolors\undefined % slides

  \definecolor{foreground}{RGB}{80,80,80}
  \definecolor{background}{RGB}{255,255,255}
  \definecolor{title}{RGB}{255,199,0}
  \definecolor{subtitle}{RGB}{89,132,212}
  \definecolor{hilit}{RGB}{248,117,79}
  \definecolor{vhilit}{RGB}{255,111,207}
  \definecolor{lolit}{RGB}{200,200,200}
  \definecolor{lit}{RGB}{255,199,0}
  \definecolor{mdlit}{RGB}{89,132,212}
  \definecolor{link}{RGB}{248,117,79}

\else % notes
  \definecolor{background}{RGB}{255,255,255}
  \definecolor{foreground}{RGB}{24,24,24}
  \definecolor{title}{RGB}{27,94,134}
  \definecolor{subtitle}{RGB}{22,175,124}
  \definecolor{hilit}{RGB}{122,0,128}
  \definecolor{vhilit}{RGB}{255,0,128}
  \definecolor{lolit}{RGB}{95,95,95}
\fi
\definecolor{nhilit}{RGB}{128,0,128}  % hilit color in notes
\definecolor{nvhilit}{RGB}{255,0,128} % vhilit for notes

\newcommand{\hilit}{\color{hilit}}
\newcommand{\vhilit}{\color{vhilit}}
\newcommand{\nhilit}{\color{nhilit}}
\newcommand{\nvhilit}{\color{nvhilit}}
\newcommand{\lit}{\color{lit}}
\newcommand{\mdlit}{\color{mdlit}}
\newcommand{\lolit}{\color{lolit}}

% use those colors
\setbeamercolor{titlelike}{fg=title}
\setbeamercolor{subtitle}{fg=subtitle}
\setbeamercolor{frametitle}{fg=gray}
\setbeamercolor{structure}{fg=subtitle}
\setbeamercolor{institute}{fg=lolit}
\setbeamercolor{normal text}{fg=foreground,bg=background}
%\setbeamercolor{item}{fg=foreground} % color of bullets
%\setbeamercolor{subitem}{fg=hilit}
%\setbeamercolor{itemize/enumerate subbody}{fg=lolit}
\setbeamertemplate{itemize subitem}{{\textendash}}
\setbeamerfont{itemize/enumerate subbody}{size=\footnotesize}
\setbeamerfont{itemize/enumerate subitem}{size=\footnotesize}

% center title of slides
\setbeamertemplate{blocks}[rounded]
\setbeamertemplate{frametitle}[default][center]
% margins
\setbeamersize{text margin left=25pt,text margin right=25pt}

% page number
\setbeamertemplate{footline}{%
    \raisebox{5pt}{\makebox[\paperwidth]{\hfill\makebox[20pt]{\lolit
          \scriptsize\insertframenumber}}}\hspace*{5pt}}

% add a bit of space at the top of the notes page
\addtobeamertemplate{note page}{\setlength{\parskip}{12pt}}

% default link color
\hypersetup{colorlinks, urlcolor={link}}

\ifx\notescolors\undefined % slides
  % set up listing environment
  \lstset{language=bash,
          basicstyle=\ttfamily\scriptsize,
          frame=single,
          commentstyle=,
          backgroundcolor=\color{darkgray},
          showspaces=false,
          showstringspaces=false
          }
\else % notes
  \lstset{language=bash,
          basicstyle=\ttfamily\scriptsize,
          frame=single,
          commentstyle=,
          backgroundcolor=\color{offwhite},
          showspaces=false,
          showstringspaces=false
          }
\fi

% a few macros
\newcommand{\code}[1]{\texttt{#1}}
\newcommand{\hicode}[1]{{\hilit \texttt{#1}}}
\newcommand{\bb}[1]{\begin{block}{#1}}
\newcommand{\eb}{\end{block}}
\newcommand{\bi}{\begin{itemize}}
%\newcommand{\bbi}{\vspace{24pt} \begin{itemize} \itemsep8pt}
\newcommand{\bbi}{\vspace{4pt} \begin{itemize} \itemsep8pt}
\newcommand{\ei}{\end{itemize}}
\newcommand{\bv}{\begin{verbatim}}
\newcommand{\ev}{\end{verbatim}}
\newcommand{\ig}{\includegraphics}
\newcommand{\subt}[1]{{\footnotesize \color{subtitle} {#1}}}
\newcommand{\ttsm}{\tt \small}
\newcommand{\ttfn}{\tt \footnotesize}
\newcommand{\figh}[2]{\centerline{\includegraphics[height=#2\textheight]{#1}}}
\newcommand{\figw}[2]{\centerline{\includegraphics[width=#2\textwidth]{#1}}}



%------------------------------------------------
% end of header
%------------------------------------------------

\title{String Basics with \code{"stringr"}}
\subtitle{STAT 133}
\author{\href{http://www.gastonsanchez.com}{Gaston Sanchez}}
\institute{Department of Statistics, UC{\textendash}Berkeley}
\date{\href{http://www.gastonsanchez.com}{\tt \scriptsize \color{foreground} gastonsanchez.com}
\\[-4pt]
\href{http://github.com/gastonstat/stat133}{\tt \scriptsize \color{foreground} github.com/gastonstat/stat133}
\\[-4pt]
{\scriptsize Course web: \href{http://www.gastonsanchez.com/stat133}{\tt gastonsanchez.com/stat133}}
}
\IfFileExists{upquote.sty}{\usepackage{upquote}}{}
\begin{document}


{
  \setbeamertemplate{footline}{} % no page number here
  \frame{
    \titlepage
  } 
}

%------------------------------------------------

\begin{frame}
\begin{center}
\Huge{\hilit{Package \code{"stringr"}}}
\end{center}
\end{frame}

%------------------------------------------------

\begin{frame}
\frametitle{About \code{"stringr"}}

\bb{About \code{"stringr"}}
\bi
  \item functions are more consistent, simpler and easier to use 
  \item \code{"stringr"} ensures that function and argument names (and positions) are consistent
  \item all functions deal with \code{NA}'s and zero length character appropriately
  \item the output data structures from each function matches the input data structures of other functions
\ei
\eb

\end{frame}

%------------------------------------------------

\begin{frame}
\frametitle{About \code{"stringr"}}

\code{"stringr"} provides functions for both:
\bi
  \item basic manipulations and,
  \item for regular expression operations. 
\ei

In this set of slides we cover those functions that have to do with basic manipulations.

\end{frame}

%------------------------------------------------

\begin{frame}[fragile]
\frametitle{About \code{"stringr"}}

\begin{knitrout}\footnotesize
\definecolor{shadecolor}{rgb}{0.969, 0.969, 0.969}\color{fgcolor}\begin{kframe}
\begin{alltt}
\hlcom{# installing 'stringr'}
\hlkwd{install.packages}\hlstd{(}\hlstr{"stringr"}\hlstd{)}

\hlcom{# load 'stringr'}
\hlkwd{library}\hlstd{(stringr)}
\end{alltt}
\end{kframe}
\end{knitrout}



\end{frame}

%------------------------------------------------

\begin{frame}
\frametitle{Basic \code{"stringr"} functions}

{\small 
\begin{center}
  \begin{tabular}{l l l}
  \hline
    Function & Description & Similar to \\
    \hline
    \code{str\_c()} & string concatenation & \code{paste()} \\
    \code{str\_length()} & number of characters & \code{nchar()} \\
    \code{str\_sub()} & extracts substrings & \code{substring()} \\
    \code{str\_dup()} & duplicates characters & \textit{none} \\
    \code{str\_trim()} & removes leading and & \textit{none} \\
      & trailing whitespace & \\
    \code{str\_pad()} & pads a string & \textit{none} \\
    \code{str\_wrap()} & wraps a string paragraph & \code{strwrap()} \\
    \code{str\_trim()} & trims a string & \textit{none} \\
    \hline
 \end{tabular}
\end{center}
}

\end{frame}

%------------------------------------------------

\begin{frame}
\frametitle{About \code{"stringr"}}

\code{stringr} provides functions for both:
\bi
  \item all functions in \code{"stringr"} start with {\hilit \code{str\_}}
  \item some functions are designed to provide a better alternative to already existing functions
  \item Other functions don't have a corresponding alternative
\ei

\end{frame}

%------------------------------------------------

\begin{frame}[fragile]
\frametitle{Function \code{str\_c()}}

\code{str\_c()} is equivalent to \code{paste()} but instead of using the white space as the default separator, \code{str\_c()} uses the empty string \code{""}
\begin{knitrout}\footnotesize
\definecolor{shadecolor}{rgb}{0.969, 0.969, 0.969}\color{fgcolor}\begin{kframe}
\begin{alltt}
\hlcom{# default usage}
\hlkwd{str_c}\hlstd{(}\hlstr{"May"}\hlstd{,} \hlstr{"The"}\hlstd{,} \hlstr{"Force"}\hlstd{,} \hlstr{"Be"}\hlstd{,} \hlstr{"With"}\hlstd{,} \hlstr{"You"}\hlstd{)}
\end{alltt}
\begin{verbatim}
## [1] "MayTheForceBeWithYou"
\end{verbatim}
\end{kframe}
\end{knitrout}

\end{frame}

%------------------------------------------------

\begin{frame}[fragile]
\frametitle{Function \code{str\_c()}}

Another major difference between \code{str\_c()} and \code{paste()}: zero length arguments like \code{NULL} and \code{character(0)} are silently removed by \code{str\_c()}.

\begin{knitrout}\footnotesize
\definecolor{shadecolor}{rgb}{0.969, 0.969, 0.969}\color{fgcolor}\begin{kframe}
\begin{alltt}
\hlcom{# removing zero length objects}
\hlkwd{str_c}\hlstd{(}\hlstr{"May"}\hlstd{,} \hlstr{"The"}\hlstd{,} \hlstr{"Force"}\hlstd{,} \hlkwa{NULL}\hlstd{,} \hlstr{"Be"}\hlstd{,} \hlstr{"With"}\hlstd{,} \hlstr{"You"}\hlstd{,}
      \hlkwd{character}\hlstd{(}\hlnum{0}\hlstd{))}
\end{alltt}
\begin{verbatim}
## [1] "MayTheForceBeWithYou"
\end{verbatim}
\end{kframe}
\end{knitrout}

\end{frame}

%------------------------------------------------

\begin{frame}[fragile]
\frametitle{Function \code{str\_c()}}

\code{str\_c()} is equivalent to \code{paste()} but instead of using the white space as the default separator, \code{str\_c()} uses the empty string \code{""}
\begin{knitrout}\footnotesize
\definecolor{shadecolor}{rgb}{0.969, 0.969, 0.969}\color{fgcolor}\begin{kframe}
\begin{alltt}
\hlcom{# changing separator}
\hlkwd{str_c}\hlstd{(}\hlstr{"May"}\hlstd{,} \hlstr{"The"}\hlstd{,} \hlstr{"Force"}\hlstd{,} \hlstr{"Be"}\hlstd{,} \hlstr{"With"}\hlstd{,} \hlstr{"You"}\hlstd{,} \hlkwc{sep}\hlstd{=}\hlstr{"_"}\hlstd{)}
\end{alltt}
\begin{verbatim}
## [1] "May_The_Force_Be_With_You"
\end{verbatim}
\begin{alltt}
\hlcom{# synonym function 'str_join'}
\hlkwd{str_join}\hlstd{(}\hlstr{"May"}\hlstd{,} \hlstr{"The"}\hlstd{,} \hlstr{"Force"}\hlstd{,} \hlstr{"Be"}\hlstd{,} \hlstr{"With"}\hlstd{,} \hlstr{"You"}\hlstd{,} \hlkwc{sep}\hlstd{=}\hlstr{"-"}\hlstd{)}
\end{alltt}


{\ttfamily\noindent\color{warningcolor}{\#\# Warning: 'str\_join' is deprecated.\\\#\# Use 'str\_c' instead.\\\#\# See help("{}Deprecated"{})}}\begin{verbatim}
## [1] "May-The-Force-Be-With-You"
\end{verbatim}
\end{kframe}
\end{knitrout}

\end{frame}

%------------------------------------------------

\begin{frame}[fragile]
\frametitle{Function \code{str\_length()}}

\code{str\_length()} is equivalent to \code{nchar()}, returning the number of characters in a string
\begin{knitrout}\footnotesize
\definecolor{shadecolor}{rgb}{0.969, 0.969, 0.969}\color{fgcolor}\begin{kframe}
\begin{alltt}
\hlcom{# some text (NA included)}
\hlstd{some_text} \hlkwb{=} \hlkwd{c}\hlstd{(}\hlstr{"one"}\hlstd{,} \hlstr{"two"}\hlstd{,} \hlstr{"three"}\hlstd{,} \hlnum{NA}\hlstd{,} \hlstr{"five"}\hlstd{)}

\hlcom{# compare 'str_length' with 'nchar'}
\hlkwd{nchar}\hlstd{(some_text)}
\end{alltt}
\begin{verbatim}
## [1] 3 3 5 2 4
\end{verbatim}
\begin{alltt}
\hlkwd{str_length}\hlstd{(some_text)}
\end{alltt}
\begin{verbatim}
## [1]  3  3  5 NA  4
\end{verbatim}
\end{kframe}
\end{knitrout}

\end{frame}

%------------------------------------------------

\begin{frame}[fragile]
\frametitle{Function \code{str\_length()}}

\code{str\_length()} has the nice feature that it converts factors to characters, something that \code{nchar()} is not able to handle:
\begin{knitrout}\footnotesize
\definecolor{shadecolor}{rgb}{0.969, 0.969, 0.969}\color{fgcolor}\begin{kframe}
\begin{alltt}
\hlcom{# some factor}
\hlstd{some_factor} \hlkwb{=} \hlkwd{factor}\hlstd{(}\hlkwd{c}\hlstd{(}\hlnum{1}\hlstd{,} \hlnum{1}\hlstd{,} \hlnum{1}\hlstd{,} \hlnum{2}\hlstd{,} \hlnum{2}\hlstd{,} \hlnum{2}\hlstd{),}
                     \hlkwc{labels} \hlstd{=} \hlkwd{c}\hlstd{(}\hlstr{"good"}\hlstd{,} \hlstr{"bad"}\hlstd{))}
\hlstd{some_factor}
\end{alltt}
\begin{verbatim}
## [1] good good good bad  bad  bad 
## Levels: good bad
\end{verbatim}
\begin{alltt}
\hlcom{# 'str_length' on a factor:}
\hlkwd{str_length}\hlstd{(some_factor)}
\end{alltt}
\begin{verbatim}
## [1] 4 4 4 3 3 3
\end{verbatim}
\end{kframe}
\end{knitrout}

\end{frame}

%------------------------------------------------

\begin{frame}[fragile]
\frametitle{Function \code{str\_length()}}

Compare \code{str\_length()} against \code{nchar()}
\begin{knitrout}\footnotesize
\definecolor{shadecolor}{rgb}{0.969, 0.969, 0.969}\color{fgcolor}\begin{kframe}
\begin{alltt}
\hlcom{# some factor}
\hlstd{some_factor} \hlkwb{=} \hlkwd{factor}\hlstd{(}\hlkwd{c}\hlstd{(}\hlnum{1}\hlstd{,}\hlnum{1}\hlstd{,}\hlnum{1}\hlstd{,}\hlnum{2}\hlstd{,}\hlnum{2}\hlstd{,}\hlnum{2}\hlstd{),}
                     \hlkwc{labels} \hlstd{=} \hlkwd{c}\hlstd{(}\hlstr{"good"}\hlstd{,} \hlstr{"bad"}\hlstd{))}

\hlcom{# now try 'nchar' on a factor}
\hlkwd{nchar}\hlstd{(some_factor)}
\end{alltt}


{\ttfamily\noindent\bfseries\color{errorcolor}{\#\# Error in nchar(some\_factor): 'nchar()' requires a character vector}}\end{kframe}
\end{knitrout}

\end{frame}

%------------------------------------------------

\begin{frame}[fragile]
\frametitle{Function \code{str\_substr()}}

\begin{knitrout}\footnotesize
\definecolor{shadecolor}{rgb}{0.969, 0.969, 0.969}\color{fgcolor}\begin{kframe}
\begin{alltt}
\hlcom{# some text}
\hlstd{lorem} \hlkwb{=} \hlstr{"Lorem Ipsum"}

\hlcom{# apply 'str_sub'}
\hlkwd{str_sub}\hlstd{(lorem,} \hlkwc{start}\hlstd{=}\hlnum{1}\hlstd{,} \hlkwc{end}\hlstd{=}\hlnum{5}\hlstd{)}
\end{alltt}
\begin{verbatim}
## [1] "Lorem"
\end{verbatim}
\begin{alltt}
\hlcom{# equivalent to 'substring'}
\hlkwd{substring}\hlstd{(lorem,} \hlkwc{first}\hlstd{=}\hlnum{1}\hlstd{,} \hlkwc{last}\hlstd{=}\hlnum{5}\hlstd{)}
\end{alltt}
\begin{verbatim}
## [1] "Lorem"
\end{verbatim}
\end{kframe}
\end{knitrout}

\end{frame}

%------------------------------------------------

\begin{frame}[fragile]
\frametitle{Function \code{str\_substr()}}

\code{str\_sub()} allows you to work with negative indices in the \code{start} and \code{end} positions:
\begin{knitrout}\footnotesize
\definecolor{shadecolor}{rgb}{0.969, 0.969, 0.969}\color{fgcolor}\begin{kframe}
\begin{alltt}
\hlcom{# some strings}
\hlstd{resto} \hlkwb{=} \hlkwd{c}\hlstd{(}\hlstr{"brasserie"}\hlstd{,} \hlstr{"bistrot"}\hlstd{,} \hlstr{"creperie"}\hlstd{,} \hlstr{"bouchon"}\hlstd{)}

\hlcom{# 'str_sub' with negative positions}
\hlkwd{str_sub}\hlstd{(resto,} \hlkwc{start}\hlstd{=}\hlopt{-}\hlnum{4}\hlstd{,} \hlkwc{end}\hlstd{=}\hlopt{-}\hlnum{1}\hlstd{)}
\end{alltt}
\begin{verbatim}
## [1] "erie" "trot" "erie" "chon"
\end{verbatim}
\end{kframe}
\end{knitrout}
When we use a negative position, \code{str\_sub()} counts backwards from last character.

\end{frame}

%------------------------------------------------

\begin{frame}[fragile]
\frametitle{Function \code{str\_sub()}}

A related function is \code{str\_sub()}; when given a set of positions they will be recycled over the string
\begin{knitrout}\footnotesize
\definecolor{shadecolor}{rgb}{0.969, 0.969, 0.969}\color{fgcolor}\begin{kframe}
\begin{alltt}
\hlcom{# extracting sequentially}
\hlkwd{str_sub}\hlstd{(lorem,} \hlkwd{seq_len}\hlstd{(}\hlkwd{nchar}\hlstd{(lorem)))}
\end{alltt}
\begin{verbatim}
##  [1] "Lorem Ipsum" "orem Ipsum"  "rem Ipsum"   "em Ipsum"    "m Ipsum"    
##  [6] " Ipsum"      "Ipsum"       "psum"        "sum"         "um"         
## [11] "m"
\end{verbatim}
\end{kframe}
\end{knitrout}

\end{frame}

%------------------------------------------------

\begin{frame}[fragile]
\frametitle{Function \code{str\_sub()}}

We can also give \code{str\_sub()} a negative sequence, something that \code{substring()} ignores:
\begin{knitrout}\footnotesize
\definecolor{shadecolor}{rgb}{0.969, 0.969, 0.969}\color{fgcolor}\begin{kframe}
\begin{alltt}
\hlcom{# reverse substrings with negative positions}
\hlkwd{str_sub}\hlstd{(lorem,} \hlopt{-}\hlkwd{seq_len}\hlstd{(}\hlkwd{nchar}\hlstd{(lorem)))}
\end{alltt}
\begin{verbatim}
##  [1] "m"           "um"          "sum"         "psum"        "Ipsum"      
##  [6] " Ipsum"      "m Ipsum"     "em Ipsum"    "rem Ipsum"   "orem Ipsum" 
## [11] "Lorem Ipsum"
\end{verbatim}
\end{kframe}
\end{knitrout}

\end{frame}

%------------------------------------------------

\begin{frame}[fragile]
\frametitle{Function \code{str\_sub()}}

We can use \code{str\_sub()} not only for extracting subtrings but also for replacing substrings:
\begin{knitrout}\footnotesize
\definecolor{shadecolor}{rgb}{0.969, 0.969, 0.969}\color{fgcolor}\begin{kframe}
\begin{alltt}
\hlcom{# replacing 'Lorem' with 'Nullam'}
\hlstd{lorem} \hlkwb{<-} \hlstr{"Lorem Ipsum"}
\hlkwd{str_sub}\hlstd{(lorem,} \hlnum{1}\hlstd{,} \hlnum{5}\hlstd{)} \hlkwb{<-} \hlstr{"Nullam"}
\hlstd{lorem}
\end{alltt}
\begin{verbatim}
## [1] "Nullam Ipsum"
\end{verbatim}
\end{kframe}
\end{knitrout}

\end{frame}

%------------------------------------------------

\begin{frame}[fragile]
\frametitle{Function \code{str\_sub()}}

\begin{knitrout}\footnotesize
\definecolor{shadecolor}{rgb}{0.969, 0.969, 0.969}\color{fgcolor}\begin{kframe}
\begin{alltt}
\hlcom{# replacing with negative positions}
\hlstd{lorem} \hlkwb{=} \hlstr{"Lorem Ipsum"}
\hlkwd{str_sub}\hlstd{(lorem,} \hlopt{-}\hlnum{1}\hlstd{)} \hlkwb{<-} \hlstr{"Nullam"}
\hlstd{lorem}
\end{alltt}
\begin{verbatim}
## [1] "Lorem IpsuNullam"
\end{verbatim}
\begin{alltt}
\hlcom{# multiple replacements }
\hlstd{lorem} \hlkwb{=} \hlstr{"Lorem Ipsum"}
\hlkwd{str_sub}\hlstd{(lorem,} \hlkwd{c}\hlstd{(}\hlnum{1}\hlstd{,}\hlnum{7}\hlstd{),} \hlkwd{c}\hlstd{(}\hlnum{5}\hlstd{,}\hlnum{8}\hlstd{))} \hlkwb{<-} \hlkwd{c}\hlstd{(}\hlstr{"Nullam"}\hlstd{,} \hlstr{"Enim"}\hlstd{)}
\hlstd{lorem}
\end{alltt}
\begin{verbatim}
## [1] "Nullam Ipsum"  "Lorem Enimsum"
\end{verbatim}
\end{kframe}
\end{knitrout}

\end{frame}

%------------------------------------------------

\begin{frame}[fragile]
\frametitle{Duplication with \code{str\_dup()}}

\code{str\_dup()} duplicates and concatenates strings within a character vector:
\begin{knitrout}\footnotesize
\definecolor{shadecolor}{rgb}{0.969, 0.969, 0.969}\color{fgcolor}\begin{kframe}
\begin{alltt}
\hlcom{# default usage}
\hlkwd{str_dup}\hlstd{(}\hlstr{"hola"}\hlstd{,} \hlnum{3}\hlstd{)}
\end{alltt}
\begin{verbatim}
## [1] "holaholahola"
\end{verbatim}
\begin{alltt}
\hlcom{# use with differetn 'times'}
\hlkwd{str_dup}\hlstd{(}\hlstr{"adios"}\hlstd{,} \hlnum{1}\hlopt{:}\hlnum{3}\hlstd{)}
\end{alltt}
\begin{verbatim}
## [1] "adios"           "adiosadios"      "adiosadiosadios"
\end{verbatim}
\end{kframe}
\end{knitrout}

\end{frame}

%------------------------------------------------

\begin{frame}[fragile]
\frametitle{Duplication with \code{str\_dup()}}

\begin{knitrout}\footnotesize
\definecolor{shadecolor}{rgb}{0.969, 0.969, 0.969}\color{fgcolor}\begin{kframe}
\begin{alltt}
\hlcom{# use with a string vector}
\hlstd{words} \hlkwb{<-} \hlkwd{c}\hlstd{(}\hlstr{"lorem"}\hlstd{,} \hlstr{"ipsum"}\hlstd{,} \hlstr{"dolor"}\hlstd{)}
\hlkwd{str_dup}\hlstd{(words,} \hlnum{2}\hlstd{)}
\end{alltt}
\begin{verbatim}
## [1] "loremlorem" "ipsumipsum" "dolordolor"
\end{verbatim}
\begin{alltt}
\hlkwd{str_dup}\hlstd{(words,} \hlnum{1}\hlopt{:}\hlnum{3}\hlstd{)}
\end{alltt}
\begin{verbatim}
## [1] "lorem"           "ipsumipsum"      "dolordolordolor"
\end{verbatim}
\end{kframe}
\end{knitrout}

\end{frame}

%------------------------------------------------

\begin{frame}[fragile]
\frametitle{Padding with \code{str\_pad()}}

Another handy function that we can find in \code{stringr} is \code{str\_pad()}  for \textit{padding} a string. Its default usage has the following form:
\begin{verbatim}
 str_pad(string, width, side = "left", pad = " ")
\end{verbatim}
The idea of \code{str\_pad()} is to take a string and pad it with leading or trailing characters to a specified total \code{width}.

\end{frame}

%------------------------------------------------

\begin{frame}[fragile]
\frametitle{Padding with \code{str\_pad()}}

\begin{knitrout}\footnotesize
\definecolor{shadecolor}{rgb}{0.969, 0.969, 0.969}\color{fgcolor}\begin{kframe}
\begin{alltt}
\hlcom{# default usage}
\hlkwd{str_pad}\hlstd{(}\hlstr{"hola"}\hlstd{,} \hlkwc{width}\hlstd{=}\hlnum{7}\hlstd{)}
\end{alltt}
\begin{verbatim}
## [1] "   hola"
\end{verbatim}
\begin{alltt}
\hlcom{# pad both sides}
\hlkwd{str_pad}\hlstd{(}\hlstr{"adios"}\hlstd{,} \hlkwc{width}\hlstd{=}\hlnum{7}\hlstd{,} \hlkwc{side}\hlstd{=}\hlstr{"both"}\hlstd{)}
\end{alltt}
\begin{verbatim}
## [1] " adios "
\end{verbatim}
\end{kframe}
\end{knitrout}

\end{frame}

%------------------------------------------------

\begin{frame}[fragile]
\frametitle{Padding with \code{str\_pad()}}

\begin{knitrout}\footnotesize
\definecolor{shadecolor}{rgb}{0.969, 0.969, 0.969}\color{fgcolor}\begin{kframe}
\begin{alltt}
\hlcom{# left padding with '#'}
\hlkwd{str_pad}\hlstd{(}\hlstr{"hashtag"}\hlstd{,} \hlkwc{width}\hlstd{=}\hlnum{8}\hlstd{,} \hlkwc{pad}\hlstd{=}\hlstr{"#"}\hlstd{)}
\end{alltt}
\begin{verbatim}
## [1] "#hashtag"
\end{verbatim}
\begin{alltt}
\hlcom{# pad both sides with '-'}
\hlkwd{str_pad}\hlstd{(}\hlstr{"hashtag"}\hlstd{,} \hlkwc{width}\hlstd{=}\hlnum{9}\hlstd{,} \hlkwc{side}\hlstd{=}\hlstr{"both"}\hlstd{,} \hlkwc{pad}\hlstd{=}\hlstr{"-"}\hlstd{)}
\end{alltt}
\begin{verbatim}
## [1] "-hashtag-"
\end{verbatim}
\end{kframe}
\end{knitrout}

\end{frame}

%------------------------------------------------

\begin{frame}[fragile]
\frametitle{Wrapping with \code{str\_wrap()}}

The function \code{str\_wrap()} is equivalent to \code{strwrap()} which can be used to \textit{wrap} a string to format paragraphs.
Its default usage has the following form:
\begin{verbatim}
 str_wrap(string, width = 80, indent = 0, exdent = 0)
\end{verbatim}

\end{frame}

%------------------------------------------------

\begin{frame}[fragile]
\frametitle{Padding with \code{str\_wrap()}}

\begin{knitrout}\footnotesize
\definecolor{shadecolor}{rgb}{0.969, 0.969, 0.969}\color{fgcolor}\begin{kframe}
\begin{alltt}
\hlcom{# quote (by Douglas Adams)}
\hlstd{some_quote} \hlkwb{<-} \hlkwd{c}\hlstd{(}
  \hlstr{"I may not have gone"}\hlstd{,}
  \hlstr{"where I intended to go,"}\hlstd{,}
  \hlstr{"but I think I have ended up"}\hlstd{,}
  \hlstr{"where I needed to be"}\hlstd{)}

\hlcom{# some_quote in a single paragraph}
\hlstd{some_quote} \hlkwb{<-} \hlkwd{paste}\hlstd{(some_quote,} \hlkwc{collapse} \hlstd{=} \hlstr{" "}\hlstd{)}
\end{alltt}
\end{kframe}
\end{knitrout}

\end{frame}

%------------------------------------------------

\begin{frame}[fragile]
\frametitle{Padding with \code{str\_wrap()}}

Say we want to display the text of \code{some\_quote} within some pre-specified column width (e.g. width of 30):
\begin{knitrout}\footnotesize
\definecolor{shadecolor}{rgb}{0.969, 0.969, 0.969}\color{fgcolor}\begin{kframe}
\begin{alltt}
\hlcom{# display paragraph with width=30}
\hlkwd{cat}\hlstd{(}\hlkwd{str_wrap}\hlstd{(some_quote,} \hlkwc{width} \hlstd{=} \hlnum{30}\hlstd{))}
\end{alltt}
\begin{verbatim}
## I may not have gone where I
## intended to go, but I think I
## have ended up where I needed
## to be
\end{verbatim}
\end{kframe}
\end{knitrout}

\end{frame}

%------------------------------------------------

\begin{frame}[fragile]
\frametitle{Trimming with \code{str\_trim()}}

One of the typical tasks of string processing is that of parsing a text into individual words.

Usually, we end up with words that have blank spaces, called \textit{whitespaces}, on either end of the word. In this situation, we can use the \code{str\_trim()} function to remove any number of whitespaces at the ends of a string. Its usage requires only two arguments:
\begin{verbatim}
 str_trim(string, side = "both")
\end{verbatim}

\end{frame}

%------------------------------------------------

\begin{frame}[fragile]
\frametitle{Padding with \code{str\_trim()}}

\begin{knitrout}\footnotesize
\definecolor{shadecolor}{rgb}{0.969, 0.969, 0.969}\color{fgcolor}\begin{kframe}
\begin{alltt}
\hlcom{# text with whitespaces}
\hlstd{bad_text} \hlkwb{<-} \hlkwd{c}\hlstd{(}\hlstr{" several   "}\hlstd{,} \hlstr{"   whitespaces "}\hlstd{)}

\hlcom{# remove whitespaces on the left side}
\hlkwd{str_trim}\hlstd{(bad_text,} \hlkwc{side} \hlstd{=} \hlstr{"left"}\hlstd{)}
\end{alltt}
\begin{verbatim}
## [1] "several   "   "whitespaces "
\end{verbatim}
\begin{alltt}
\hlcom{# remove whitespaces on the right side}
\hlkwd{str_trim}\hlstd{(bad_text,} \hlkwc{side} \hlstd{=} \hlstr{"right"}\hlstd{)}
\end{alltt}
\begin{verbatim}
## [1] " several"       "   whitespaces"
\end{verbatim}
\begin{alltt}
\hlcom{# remove whitespaces on both sides}
\hlkwd{str_trim}\hlstd{(bad_text,} \hlkwc{side} \hlstd{=} \hlstr{"both"}\hlstd{)}
\end{alltt}
\begin{verbatim}
## [1] "several"     "whitespaces"
\end{verbatim}
\end{kframe}
\end{knitrout}

\end{frame}

%------------------------------------------------

\begin{frame}[fragile]
\frametitle{Word extraction with \code{word()}}

\code{word()} function that is designed to extract words from a sentence:
\begin{verbatim}
 word(string, start = 1L, end = start, sep = fixed(" "))
\end{verbatim}

The way in which we use \code{word()} is by passing it a \code{string}, together with a \code{start} position of the first word to extract, and an \code{end} position of the last word to extract. By default, the separator \code{sep} used between words is a single space.

\end{frame}

%------------------------------------------------

\begin{frame}[fragile]
\frametitle{Word extraction with \code{word()}}

\begin{knitrout}\footnotesize
\definecolor{shadecolor}{rgb}{0.969, 0.969, 0.969}\color{fgcolor}\begin{kframe}
\begin{alltt}
\hlcom{# some sentence}
\hlstd{change} \hlkwb{=} \hlkwd{c}\hlstd{(}\hlstr{"Be the change"}\hlstd{,} \hlstr{"you want to be"}\hlstd{)}

\hlcom{# extract first word}
\hlkwd{word}\hlstd{(change,} \hlnum{1}\hlstd{)}
\end{alltt}
\begin{verbatim}
## [1] "Be"  "you"
\end{verbatim}
\begin{alltt}
\hlcom{# extract second word}
\hlkwd{word}\hlstd{(change,} \hlnum{2}\hlstd{)}
\end{alltt}
\begin{verbatim}
## [1] "the"  "want"
\end{verbatim}
\end{kframe}
\end{knitrout}

\end{frame}

%------------------------------------------------

\begin{frame}[fragile]
\frametitle{Word extraction with \code{word()}}

\begin{knitrout}\footnotesize
\definecolor{shadecolor}{rgb}{0.969, 0.969, 0.969}\color{fgcolor}\begin{kframe}
\begin{alltt}
\hlcom{# some sentence}
\hlstd{change} \hlkwb{=} \hlkwd{c}\hlstd{(}\hlstr{"Be the change"}\hlstd{,} \hlstr{"you want to be"}\hlstd{)}

\hlcom{# extract last word}
\hlkwd{word}\hlstd{(change,} \hlopt{-}\hlnum{1}\hlstd{)}
\end{alltt}
\begin{verbatim}
## [1] "change" "be"
\end{verbatim}
\begin{alltt}
\hlcom{# extract all but the first words}
\hlkwd{word}\hlstd{(change,} \hlnum{2}\hlstd{,} \hlopt{-}\hlnum{1}\hlstd{)}
\end{alltt}
\begin{verbatim}
## [1] "the change" "want to be"
\end{verbatim}
\end{kframe}
\end{knitrout}

\end{frame}

%------------------------------------------------


\end{document}
